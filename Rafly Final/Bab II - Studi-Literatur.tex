% ==========================================
% BAB II STUDI LITERATUR
% ==========================================
\chapter{STUDI LITERATUR}
\label{chap:studi-literatur}
\section{Meteorologi Petir dan Ground Flash Density (GFD)}

\subsection{Konsep Dasar dan Metrik GFD}
Pengukuran dan pemodelan \textit{Ground Flash Density} (GFD) sangat bergantung pada ketersediaan data petir yang andal dari pengamatan satelit. Sensor satelit seperti \textit{Lightning Imaging Sensor} (LIS) dan \textit{Optical Transient Detector} (OTD) mencatat petir sebagai \textit"{events}" pada array pencitraan \textit{Charge-Coupled Device} (CCD), yang merepresentasikan peristiwa optik mentah akibat kilatan cahaya petir. Untuk mengidentifikasi sambaran petir yang sesungguhnya, data events ini harus melalui tahap pemrosesan tambahan berupa \textit{filtering} dan \textit{clustering}, sehingga diperoleh struktur data bertingkat berupa \textit{groups}, \textit{flashes}, dan \textit{areas} \cite{peterson2021LISOTD}.

Dalam pemrosesan standar LIS/OTD, \textit{groups} didefinisikan sebagai klaster iluminasi puncak awan dari pulsa-pulsa cahaya individu, \textit{flashes} didefinisikan sebagai rangkaian pulsa yang menggambarkan satu kilat yang utuh, sedangkan \textit{areas} merepresentasikan klaster aktivitas petir yang menggambarkan \textit{snapshot} badai. Dalam konteks GFD, penting untuk membedakan antara \textit{events}, \textit{groups}, dan \textit{flashes} karena GFD didefinisikan sebagai jumlah sambaran petir ke darat (\textit{ground flashes}) per satuan luas dan waktu (misalnya per km² per tahun), sehingga skema klasifikasi ini menjadi dasar untuk menghitung densitas sambaran yang menjadi target prediksi. LIS/OTD berfungsi sebagai salah satu sumber \textit{ground truth} global utama yang relevan untuk memvalidasi model-model prediksi GFD, termasuk model berbasis komputasi kuantum yang dikembangkan pada tesis ini \cite{peterson2021LISOTD}.

\subsection{Faktor Meteorologis Pembentuk Petir}
Pembentukan petir sangat dipengaruhi oleh kondisi termodinamika dan dinamika atmosfer yang dapat direpresentasikan melalui variabel-variabel meteorologis skala besar. Pada sebuah studi parameterisasi petir berbasis ML, himpunan data pelatihan dirangkum dalam Tabel \ref{tbl:tabelVariabelCheng} dengan variabel input yang mencakup presipitasi ($P$) dan berbagai variabel medan atmosfer, seperti \textit{convective available potential energy} (CAPE), ketinggian geopotensial 500 hPa (Z500), ketinggian geopotensial 1000 hPa (Z1000), ketebalan antara level 300–700 hPa (Z300–700), suhu pada 2-m (T2M), kecepatan vertikal pada 500 hPa, serta suhu titik embun pada 2-m (2-m \textit{dewpoint temperature}) \cite{cheng2024MLLP}.

\input table/tabelVariabelCheng.tex

CAPE pada studi tersebut digunakan sebagai indikator utama ketidakstabilan atmosfer yang mendukung konveksi kuat, sedangkan presipitasi (P) dan suhu titik embun 2m secara langsung terkait dengan ketersediaan uap air dan kelembaban yang menjadi prasyarat pembentukan awan konvektif dan petir. Variabel-variabel atmosfer skala besar ini terbukti penting dalam parameterisasi petir dan, dalam konteks tesis ini, dapat difungsikan sebagai fitur input klasik untuk model \textit{Quantum Neural Network} (QNN) yang dikembangkan untuk memprediksi GFD \cite{cheng2024MLLP}.

\subsection{Disparitas Karakteristik: Tropis Versus Subtropis}
Indonesia digambarkan sebagai negara tropis yang terletak di sekitar ekuator dan dikelilingi oleh lautan, sehingga mengalami cuaca panas sepanjang tahun. Karakter maritim dengan banyak pulau ini menghasilkan aerosol dalam jumlah besar, terutama dari garam laut, sedangkan iklim yang lembab dengan hutan, sungai, dan danau yang melimpah menjadi faktor utama pembentukan awan kumulonimbus (CB) sebagai awan penghasil petir. Kondisi-kondisi ini menjadikan wilayah tropis maritim seperti Indonesia sangat kondusif terhadap aktivitas petir yang intensif \cite{denov2023}.

Studi lain menegaskan bahwa terdapat disparitas yang signifikan antara perilaku petir di iklim tropis dan subtropis \cite{denov2025RNFPA}. Investigasi ekstensif selama delapan tahun di Stasiun Penelitian Petir Gunung Tangkuban Perahu menemukan berbagai karakteristik unik petir tropis, menandakan bahwa karakteristik kilat di wilayah tropis tidak dapat secara langsung disamakan dengan petir di wilayah subtropis \cite{denov2023}. Perbedaan ini menjustifikasi perlunya model prediksi yang secara eksplisit mempertimbangkan perbedaan iklim tropis dan subtropis, terutama ketika standar proteksi petir internasional yang dominan berbasis pada karakteristik subtropis hendak diterapkan di wilayah tropis \cite{denov2025RNFPA}.

\section{Pendekatan Machine Learning Klasik}

\subsection{Algoritma \textit{Deep Learning} (CNN/LSTM) untuk Nowcasting}
Dalam ranah prediksi petir berbasis komputasi klasik, DL telah mencapai tingkat kompleksitas arsitektur yang tinggi. Salah satu contoh adalah model LightningNet yang menggunakan arsitektur \textit{encoder–decoder} dengan 20 lapisan konvolusional tiga dimensi, lapisan \textit{pooling} dan \textit{upsampling}, lapisan normalisasi, serta \textit{softmax classifier}. Arsitektur ini dirancang untuk memproses data multi-sumber dalam format spasial atau citra, seperti data satelit, radar, dan lokasi petir, sehingga mampu melakukan \textit{nowcasting} petir cloud-to-ground (CG) dengan resolusi yang tinggi. Model-model seperti LightningNet menunjukkan bahwa prediksi petir modern menuntut pemrosesan informasi spasial yang sangat kompleks, yang menjadi tolok ukur kinerja bagi model QNN yang akan dikembangkan \cite{zhou2020DLN}.

Di sisi lain, RNN dirancang khusus untuk memproses data urutan (\textit{sequence data}) dengan memprosesnya secara iteratif dalam arah sekuensial. Dalam konteks prediksi petir, RNN sering direalisasikan sebagai \textit{long short-term memory} (LSTM) yang dikembangkan untuk mengatasi masalah ketergantungan jangka panjang yang sulit ditangani oleh RNN standar. Prediksi GFD yang memanfaatkan deret waktu variabel-variabel meteorologis seperti CAPE dan kelembaban dari waktu ke waktu sangat relevan dengan kemampuan LSTM dalam menangkap ketergantungan temporal jangka panjang. Kombinasi CNN dan LSTM menghasilkan arsitektur ConvLSTM yang lazim digunakan untuk prediksi \textit{spatiotemporal} dalam meteorologi, dan fakta ini menandakan bahwa QNN yang dikembangkan dalam penelitian ini juga harus mampu menangani dependensi temporal secara efektif \cite{zhou2020DLN}.

\subsection{Keterbatasan Komputasi Klasik}
Penggunaan \textit{Earth Observation} (EO) dan data cuaca dalam jumlah besar untuk prediksi cuaca ekstrem menyebabkan kompleksitas waktu komputasi yang sangat tinggi pada pendekatan komputasi klasik \cite{munasinghe2024}. Studi deep learning untuk prediksi petir, seperti LightningNet, menunjukkan bahwa model tersebut memiliki parameter yang dapat dilatih dalam jumlah sangat besar, misalnya hingga 29.128.577 parameter, sehingga pelatihan dan inferensi menjadi sangat mahal secara komputasi \cite{zhou2020DLN}. Volume data yang besar dikombinasikan dengan kebutuhan akurasi tinggi membatasi penerapan model klasik skala penuh, dan hal ini membuka peluang bagi \textit{quantum machine learning} (QML) yang menjanjikan pemrosesan data besar secara paralel dengan kecepatan dan presisi yang lebih baik \cite{munasinghe2024}.

Selain itu, atmosfer digambarkan sebagai sistem dinamis dan kacau (\textit{dynamic and chaotic system}), dengan variasi kecil pada kondisi awal dapat menyebabkan perubahan signifikan pada kondisi di masa depan. Sifat \textit{chaotic} dan kompleksitas nonlinear ini menyulitkan model klasik untuk memberikan prediksi yang akurat, terutama pada rentang waktu lebih panjang atau untuk peristiwa ekstrem seperti GFD tinggi. Meskipun CNN dan LSTM mampu memodelkan sistem nonlinear yang kompleks, generalisasi terhadap data yang tidak seimbang atau jarang terjadi, seperti sambaran petir ekstrem, tetap menjadi tantangan nyata bagi DL klasik \cite{silva2025}.

Studi lain menunjukkan bahwa ketika badai petir disimulasikan pada skala yang lebih kecil dengan resolusi spasial dan temporal yang lebih halus, prediktabilitas kejadiannya justru berkurang. Akurasi prediksi cenderung menurun seiring dengan peningkatan resolusi karena ketidakpastian prakiraan numerik (NWP) yang meningkat, sehingga terdapat batas fisik pada prediktabilitas sistem atmosfer \cite{vahid2024MLT}. Dalam konteks ini, QML diharapkan dapat mengimbangi peningkatan ketidakpastian melalui keunggulan komputasi dan representasi ruang fitur berdimensi tinggi \cite{munasinghe2024}.

\section{Komputasi Kuantum}

Komputasi kuantum adalah cara komputasi modern yang didasarkan pada ilmu mekanika kuantum dan fenomena luar biasa yang ditimbulkannya. Ini adalah kombinasi antara fisika, matematika, ilmu komputer, dan teori informasi, dan merupakan jenis komputasi baru yang berurusan dengan dunia fisik yang bersifat probabilistik dan tidak dapat diprediksi. Komputasi kuantum memanfaatkan tiga sifat mendasar mekanika kuantum superposisi, \textit{entanglement} (keterikatan), dan \textit{interference} (interferensi) untuk menyimpan, merepresentasikan, dan melakukan operasi pada data. Untuk menyimpan dan memanipulasi informasi, sistem ini menggunakan quantum bits atau \textit{qubits}, yang merupakan unit fundamental informasi kuantum dan merepresentasikan partikel subatom seperti atom, elektron, atau foton. Berbeda dengan bit klasik yang hanya memiliki nilai 0 atau 1 secara individual, \textit{qubit} dapat memiliki nilai 0, 1, atau keduanya secara bersamaan karena sifat superposisi. Dengan mengendalikan perilaku objek fisik kecil (partikel mikroskopis), komputasi kuantum dapat memberikan daya komputasi yang tinggi dan kecepatan eksponensial yang jauh melampaui komputer klasik \cite{marella2020}.

\section{Algoritma \textit{Surpervised Quantum Machine Learning} (QML)}

\subsection{\textit{Quantum Generative Adversarial Networks} (QGAN)}

Arsitektur QGAN yang diusulkan mensintesis pemodelan generatif klasik dengan prinsip \textit{quantum computing} dengan memetakan permainan \textit{adversarial minimax} ke dalam \textit{Hilbert space} $\mathcal{H}$. Dalam kerangka kerja ini, \textit{generator} dan \textit{discriminator} konvensional digantikan oleh \textit{quantum operators}, $\hat{U}_G$ dan $\hat{U}_D$, yang mengevolusi \textit{quantum states} melalui \textit{unitary transformations}, bukan lapisan \textit{neural network} klasik. Arsitektur ini bekerja dengan mengevolusi \textit{initial quantum state} $\hat{\rho}_0$ menuju distribusi target yang mengaproksimasi data yang sebenarnya. Desain ini memperkenalkan \textit{terms} khusus berbasis kuantum, $\hat{V}_G$ dan $\hat{V}_D$, ke dalam \textit{Hamiltonians} milik \textit{generator} dan \textit{discriminator}. Augmentasi ini menciptakan sinergi \textit{hybrid quantum-classical} yang bertujuan untuk mengeksploitasi \textit{superposition} dan \textit{entanglement} guna mempercepat proses evolusi \textit{state}, yang secara fundamental mendefinisikan ulang dinamika \textit{training} dibandingkan dengan \textit{baseline} klasik \cite{nokhwal2023}.

Formulasi matematis inti bergantung pada representasi \textit{density matrix} untuk mendefinisikan \textit{cost function}. Sebagaimana disempurnakan dalam Persamaan \ref{eq:objectiveQGAN}, \textit{objective function} yang ditingkatkan secara kuantum dinyatakan sebagai berikut \cite{nokhwal2023}.

\begin{equation}
\min_{\hat{U}_G} \max_{\hat{U}_D} \text{Tr}(\hat{\rho}_r \hat{U}_D) + \text{Tr}(\hat{\rho}_g \hat{U}_G \hat{U}_D)
\label{eq:objectiveQGAN}
\end{equation}

$\text{Tr}$ menunjukkan operasi \textit{trace}, sedangkan $\hat{\rho}_r$ dan $\hat{\rho}_g$ masing-masing merepresentasikan \textit{density matrices} dari \textit{quantum states} asli dan yang dibangkitkan. Komponen penting dari kerangka kerja ini adalah strategi \textit{quantum data encoding} yang fitur data klasik $p_i$ dipetakan ke dalam \textit{quantum basis states} $\ket{\psi_i}$. Evolusi dari \textit{quantum operators} diatur oleh simulasi \textit{Hamiltonian}; secara khusus, evolusi \textit{unitary} dari \textit{generator} yang diaugmentasi didefinisikan dalam Persamaan \ref{eq:quantumGeneratorQGAN} sebagai berikut \cite{nokhwal2023}.

\begin{equation}
\hat{U}_G = e^{-i(\hat{H}_G+\lambda\hat{V}_G)t} 
\label{eq:quantumGeneratorQGAN}
\end{equation}

Di sini, $\lambda$ adalah parameter kontrol yang memodulasi pengaruh \textit{quantum-enhanced term} $\hat{V}_G$ untuk mengoptimalkan lintasan konvergensi \textit{training}.

Pengaturan eksperimental yang dijelaskan dalam makalah ini bersifat teoritis, menguraikan peta jalan untuk validasi masa depan pada prosesor kuantum dengan $n$ \textit{qubits}. Implementasi teknik QGAN yang diusulkan bersifat prospektif karena keterbatasan perangkat keras saat ini. Studi ini mengusulkan penilaian kinerja menggunakan metrik generatif standar seperti \textit{Inception Score} dan \textit{Fréchet Inception Distance} (FID) pada pelaksanaan di masa mendatang. Hasil yang diantisipasi, menunjukkan bahwa integrasi mekanisme \textit{quantum speedup} berbasis \textit{Hamiltonian} akan secara signifikan mengurangi jumlah iterasi yang diperlukan untuk konvergensi dibandingkan dengan metode klasik, meskipun data empiris yang mendukung peningkatan efisiensi spesifik ini belum dihasilkan \cite{nokhwal2023}.

\subsection{\textit{Quantum Neural Network} (QNN)}

\textit{Quantum Neural Network} (QNN) adalah sebuah arsitektur \textit{feed-forward} yang dirancang untuk meniru logika struktural \textit{neural networks} klasik sambil beroperasi sepenuhnya dalam kerangka kerja mekanika kuantum. Sebagaimana diilustrasikan dalam Gambar \ref{gambar:networkQNN}, jaringan ini terdiri dari $L+2$ \textit{layer} \textit{qubit}, yang disusun dari \textit{input layer} hingga \textit{output layer}, dengan \textit{hidden layers} di antaranya \cite{beer2022}.

\begin{figure}[h]
	\centering
  \captionsetup{justification=centering}
    	\includegraphics[width=0.7\textwidth]{image/networkQNN.png}
	\caption{Jaringan QNN \cite{beer2022}}
	\label{gambar:networkQNN}
\end{figure}

Blok pembangun fundamental dari arsitektur ini adalah \textit{quantum perceptron}, yang digambarkan dalam Gambar \ref{gambar:perceptronQNN}, yang menghubungkan \textit{neurons} (\textit{qubit}) antara \textit{layer} $l-1$ dan \textit{layer} $l$. \textit{Perceptron} ini diimplementasikan sebagai operasi \textit{unitary} yang bekerja pada \textit{qubit} di \textit{layer} saat ini serta \textit{qubit} baru di \textit{layer} berikutnya yang diinisialisasi dalam \textit{state} $\ket{0}$, diikuti dengan operasi \textit{tracing out} pada \textit{qubit} dari \textit{layer} sebelumnya. Operasi \textit{partial trace} ini menjadikan transisi antar-\textit{layer} bersifat disipatif, yang mencegah pembalikan aliran informasi dan secara formal menstrukturkan jaringan sebagai rantai interaksi lokal alih-alih satu \textit{unitary} global tunggal. Desain sirkuit yang dihasilkan memungkinkan pengurangan kebutuhan memori selama \textit{training}, karena optimasi \textit{perceptron} tertentu hanya memerlukan akses ke \textit{quantum states} dari \textit{layer} tetangga terdekatnya \cite{beer2022}.

\begin{figure}[h]
	\centering
  \captionsetup{justification=centering}
    	\includegraphics[width=0.7\textwidth]{image/perceptronQNN.png}
	\caption{\textit{Quantum perceptron} \cite{beer2022}}
	\label{gambar:perceptronQNN}
\end{figure}

Formulasi matematis memperlakukan QNN sebagai komposisi dari \textit{layer-to-layer completely positive} (CP) \textit{maps}. Propagasi sebuah \textit{input state} $\rho^{\text{in}}$ melalui jaringan menuju \textit{output} $\rho^{\text{out}}$ didefinisikan dalam Persamaan \ref{eq:outputQNN} sebagai berikut \cite{beer2022}.

\begin{equation}
\rho^{\text{out}} = \mathcal{E}(\rho^{\text{in}}) = \mathcal{E}^{L+1}(\dots\mathcal{E}^1(\rho^{\text{in}})\dots)
\label{eq:outputQNN}
\end{equation}

Setiap $\mathcal{E}^l$ merepresentasikan peta transisi. Proses \textit{training} bergantung pada \textit{supervised learning} yang menggunakan pasangan \textit{input} dan \textit{output quantum states} yang diinginkan. \textit{Cost function} didefinisikan sebagai \textit{fidelity} antara \textit{output} jaringan dan target \textit{output}, yang dirata-ratakan pada \textit{training set}. Untuk mengoptimalkan jaringan, penulis menurunkan analog kuantum terhadap algoritma \textit{back-propagation}. Parameter \textit{unitary} diperbarui melalui matriks $K_j^l(t)$ yang diturunkan. Matriks pembaruan ini memerlukan penentuan \textit{adjoint channel} $\mathcal{F}_t^l$ untuk melakukan \textit{back-propagate error} dari \textit{output layer}, yang secara efektif membentuk mekanisme \textit{gradient descent} yang beroperasi langsung pada \textit{unitary manifold} dari \textit{perceptrons} \cite{beer2022}.

Evaluasi eksperimental berfokus pada tugas mengkarakterisasi operasi \textit{unitary} yang tidak diketahui menggunakan simulasi klasik dan eksekusi pada perangkat \textit{NISQ}. Dalam simulasi klasik, QNN menunjukkan konvergensi yang cepat, dengan \textit{training} dan \textit{validation losses} mendekati \textit{fidelity} 1 dalam 1000 \textit{epochs} untuk sejumlah kecil pasangan \textit{training}. Jaringan mempertahankan \textit{validation fidelities} yang tinggi bahkan ketika \textit{noise} yang signifikan dimasukkan ke dalam data \textit{training}. Selanjutnya, perluasan model menggunakan data terstruktur dan \textit{generative adversarial training} menunjukkan bahwa arsitektur ini dapat diadaptasi untuk meningkatkan generalisasi dan menghasilkan data kuantum \cite{beer2022}.

\subsection{\textit{Quantum Support Vector Regression} (QSVR)}

Model \textit{Quantum Support Vector Regression} (QSVR) \textit{semi-supervised} yang dirancang khusus untuk \textit{anomaly detection} (AD) pada perangkat \textit{Noisy Intermediate-Scale Quantum} (NISQ). Inti dari arsitektur ini adalah metode \textit{quantum kernel} yang menghitung kesamaan antar titik data dalam \textit{feature space} berdimensi tinggi. Seperti diilustrasikan dalam Gambar \ref{gambar:sirkuitQSVR}, estimasi \textit{kernel} diimplementasikan melalui sirkuit \textit{inversion test} (atau \textit{overlap test}). Sirkuit ini menerapkan \textit{unitary feature map} $U(x_i)$ yang meng-\textit{encode} titik data pertama, diikuti segera oleh \textit{adjoint unitary} $U^\dagger(x_j)$ yang meng-\textit{encode} titik data kedua. \textit{Ansatz} spesifik untuk $U$ terdiri dari satu \textit{layer} \textit{gate} $R_Z$ dan satu \textit{layer} \textit{gate} $R_X$ untuk meng-\textit{encode} fitur individu, diikuti oleh satu \textit{layer} \textit{gate} IsingZZ untuk menghasilkan \textit{entanglement} antar \textit{qubit}. Nilai \textit{kernel} $K_{ij}$ diekstraksi dengan mengukur probabilitas \textit{quantum register} kembali ke \textit{state} nol-semua $\ket{0\dots0}$ setelah rangkaian operasi tersebut \cite{tscharke2024}.

\begin{figure}[h]
	\centering
  \captionsetup{justification=centering}
    	\includegraphics[width=1\textwidth]{image/sirkuitQSVR.png}
	\caption{Sirkuit QSVR \cite{tscharke2024}}
	\label{gambar:sirkuitQSVR}
\end{figure}

Kerangka teoretis bergantung pada pemetaan data klasik $x$ ke dalam \textit{Hilbert space} $\mathcal{F}$ melalui \textit{feature map} Persamaan \ref{eq:featureQSVR}. Studi ini menggunakan \textit{angle encoding}, sebuah bentuk \textit{time-evolution encoding} yang dijelaskan oleh evolusi uniter dalam Persamaan \ref{eq:uniterQSVR}, dengan \textit{Hamiltonian} $H$ menggunakan matriks Pauli. \textit{Quantum kernel} didefinisikan secara ketat sebagai \textit{fidelity} antara dua \textit{quantum states} yang di-\textit{encode}, yang dinyatakan secara matematis dalam Persamaan \ref{eq:fidelityQSVR} \cite{tscharke2024}.

\begin{equation}
\phi: \mathcal{X} \rightarrow \mathcal{F}
\label{eq:featureQSVR}
\end{equation}

\begin{equation}
U(x) = e^{-ixH}
\label{eq:uniterQSVR}
\end{equation}

\begin{equation}
\kappa(x_i, x_j) = |\langle \phi(x_i) | \phi(x_j) \rangle|^2
\label{eq:fidelityQSVR}
\end{equation}

Untuk menganalisis \textit{robustness}, penulis memodelkan lingkungan perangkat keras menggunakan representasi \textit{operator-sum} dari \textit{quantum channels}. Mereka memformalkan interaksi \textit{noise} spesifik secara matematis, seperti \textit{Amplitude Damping} dan \textit{Phase Damping}, untuk mensimulasikan degradasi matriks \textit{kernel} di bawah kondisi NISQ yang realistis \cite{tscharke2024}.

Validasi eksperimental melakukan \textit{benchmark} QSVR pada IBM Quantum System One 27-\textit{qubit} berbasis superkonduktor terhadap model kuantum yang disimulasikan dan \textit{baseline} klasik (CSVR, Autoencoder) di sebelas dataset. Model berbasis perangkat keras (qc-QSVR) menunjukkan performa kompetitif, yang secara mengejutkan mengungguli simulasi tanpa \textit{noise} (\textit{noiseless simulation}) pada dataset Credit Card (CC) dan KDD, menunjukkan adanya efek regularisasi yang menguntungkan dari \textit{hardware noise}. Namun, simulasi \textit{noise} ekstensif mengungkapkan bahwa meskipun model tersebut \textit{robust} terhadap \textit{depolarizing} dan \textit{bit-flip errors}, model ini sangat sensitif terhadap \textit{amplitude damping} dan \textit{miscalibration noise} \cite{tscharke2024}.

\subsection{\textit{Quantum $K$ Nearest Neighbors} (QkNN)}

Arsitektur QkNN yang diusulkan bekerja dengan mereduksi pencarian \textit{nearest neighbor} menjadi sebuah instansi dari algoritma \textit{quantum $k$ maxima finding}. Inti dari desain sirkuit ini adalah konstruksi \textit{oracle} spesifik, $\mathcal{O}_{y,A}$, yang menandai indeks dari \textit{training states} yang memiliki \textit{similarity} lebih tinggi terhadap \textit{test state} dibandingkan dengan \textit{threshold index} $y$ yang dipilih secara acak. Proses ini bergantung pada \textit{quantum subroutine} modular yang melibatkan tiga tahap: \textit{analog encoding}, konversi analog-ke-digital, dan komparasi. Pertama, metrik \textit{similarity} di-\textit{encode} ke dalam amplitudo menggunakan sirkuit \textit{Swap Test} pada Gambar \ref{gambar:swapQkNN} untuk \textit{fidelity} atau \textit{Hadamard Test} pada Gambar \ref{gambar:hadamardQkNN} untuk \textit{dot products}. Amplitudo ini kemudian didigitalkan menggunakan skema \textit{Quantum Analog-to-Digital Conversion} (QADC) yang dimodifikasi dan menggunakan \textit{Quantum Phase Estimation} (QPE) untuk meregistrasi nilai \textit{similarity} ke dalam \textit{computational basis}. Terakhir, pencarian berbasis \textit{Grover} mengiterasi struktur ini untuk mengidentifikasi dan mengganti indeks dalam himpunan $A$ hingga $k$ \textit{nearest neighbors} terisolasi \cite{basheer2024}.

\begin{figure}[h]
	\centering
  \captionsetup{justification=centering}
    	\includegraphics[width=0.7\textwidth]{image/swapQkNN.png}
	\caption{Sirkuit \textit{Swap Test} \cite{basheer2024}}
	\label{gambar:swapQkNN}
\end{figure}

\begin{figure}[h]
	\centering
  \captionsetup{justification=centering}
    	\includegraphics[width=0.7\textwidth]{image/hadamardQkNN.png}
	\caption{Sirkuit \textit{Hadamard Test} \cite{basheer2024}}
	\label{gambar:hadamardQkNN}
\end{figure}

Formulasi matematis berpusat pada pendefinisian ukuran \textit{similarity}---baik berupa \textit{fidelity} sebagaimana didefinisikan dalam \ref{eq:fidelityQkNN}, atau \textit{dot product} sebagaimana didefinisikan dalam \ref{eq:dotQkNN}. Rumus dirancang sedemikian rupa sehingga \textit{eigenstates}-nya meng-\textit{encode} informasi \textit{fidelity} dalam \textit{eigenvalues}, dengan hubungan antara fase dan \textit{fidelity}. Strategi \textit{encoding} ini memungkinkan algoritma untuk melakukan perbandingan aritmatika pada \textit{quantum states} tanpa memerlukan deskripsi klasik, yang secara fundamental bergantung pada konversi digital fase $\theta_j$ yang diekstraksi melalui QPE \cite{basheer2024}.

\begin{equation}
F(\psi, \phi_j) = |\langle\psi|\phi_j\rangle|^2
\label{eq:fidelityQkNN}
\end{equation}

\begin{equation}
X(v, u_j)
\label{eq:dotQkNN}
\end{equation}

Algoritma ini divalidasi dengan menerapkannya pada klasifikasi \textit{entanglement} dan diskriminasi \textit{quantum state}, dengan mengajukan \textit{query complexity} sebesar $O(\sqrt{kM})$ dengan $M$ sebagai jumlah \textit{training states}. Ini merepresentasikan \textit{quadratic speedup} dibandingkan dengan pencarian klasik yang menyeluruh (\textit{exhaustive search}). Simulasi numerik dilakukan untuk mengklasifikasikan \textit{n-qubit states} ke dalam kelas \textit{entanglement} yang berbeda (misalnya, \textit{separable} melawan \textit{entangled}). QkNN yang diusulkan ini sangat berbeda karena beroperasi langsung pada data kuantum melalui sirkuit \textit{state preparation}, sehingga melewati proses \textit{quantum state tomography}, yaitu sebuah proses mahal yang biasanya diperlukan oleh kNN klasik atau varian kuantum lain yang bergantung pada deskripsi \textit{state} klasik secara eksplisit \cite{basheer2024}.

\subsection{\textit{Quantum Decision Tree} (QDT)}

Metode \textit{Quantum Decision Tree} merupakan kerangka kerja klasifikasi \textit{ansatz-free} yang berbeda dari \textit{Variational Quantum Algorithms} (VQAs) tradisional karena tidak memerlukan \textit{loops} optimisasi sirkuit yang terparameterisasi. Sebaliknya, arsitektur ini menggunakan logika \textit{decision-tree} klasik dengan setiap \textit{nodes} merepresentasikan pengukuran kuantum fisik pada \textit{test state} spesifik. Seperti diilustrasikan pada \ref{gambar:algoritmaQDT}, proses dimulai dengan mempersiapkan himpunan \textit{candidate states} dan melakukan pra-komputasi \textit{expectation values} untuk sekumpulan \textit{observables} yang tersedia. Mekanisme inti beroperasi sebagai \textit{greedy search} dengan setiap langkah, algoritma memilih \textit{observable} spesifik $\mathcal{O}_j$ yang memaksimalkan \textit{expected information gain}, melakukan pengukuran \textit{single-shot}, dan memperbarui distribusi probabilitas pada kelas kandidat secara klasik. Struktur \textit{hybrid} ini mengisolasi sumber daya kuantum secara eksklusif untuk akuisisi data (\textit{state preparation} dan pengukuran), sementara logika inferensi tetap sepenuhnya klasik, sehingga menghindari kompleksitas pelatihan yang umumnya terkait dengan \textit{Quantum Neural Networks} (QNNs) \cite{li2025}.

\begin{figure}[h]
	\centering
  \captionsetup{justification=centering}
    	\includegraphics[width=1\textwidth]{image/algoritmaQDT.png}
	\caption{Algoritma QDT \cite{li2025}}
	\label{gambar:algoritmaQDT}
\end{figure}

Landasan matematis algoritma ini bergantung pada minimalisasi \textit{Shannon entropy} dari distribusi \textit{candidate state} melalui \textit{Bayesian inference}. Setelah mengukur sebuah \textit{observable} $\mathcal{O}_j$ dengan hasil $\pm 1$, probabilitas $p(i)$ bahwa \textit{test state} tersebut berkorespondensi dengan kandidat ke-$i$ diperbarui menurut \ref{eq:fidelityQkNN}, yang memanfaatkan probabilitas kondisional yang diturunkan dari \textit{expectation value} kandidat yang telah dihitung sebelumnya $\langle \mathcal{O}_j \rangle_i$. Metrik optimisasi yang digunakan adalah \textit{expected information gain}, didefinisikan sebagai selisih antara \textit{prior entropy} $H_0$ dan \textit{expected posterior entropy} $H(\mathcal{O}_j)$. Derivasi ini membuktikan secara matematis bahwa untuk \textit{Haar-random states} yang tidak terstruktur, \textit{information gain} mengalami supresi eksponensial seiring dengan ukuran sistem $N$ (berskala $1/2^n$), sebuah fenomena yang secara struktural mirip dengan masalah \textit{barren plateau} dalam \textit{gradient-based quantum learning} \cite{li2025}.

\begin{equation}
p_j(i | \pm 1) = \frac{p(i)(1 \pm \langle \mathcal{O}_j \rangle_i)}{\sum_{k=0}^{N} p(k)(1 \pm \langle \mathcal{O}_j \rangle_k)}
\label{eq:fidelityQkNN}
\end{equation}

Validasi eksperimental dilakukan menggunakan simulasi 10-\textit{qubit} dan prosesor kuantum IBM Kawasaki untuk menilai efisiensi \textit{classifier}. Strategi yang dioptimalkan informasinya (\textit{information-optimized}) mencapai konvergensi pada tingkat kepercayaan 99\% (\textit{p-value} $< 0.01$) dalam waktu sekitar 150 \textit{shots}, secara signifikan mengungguli protokol pengukuran acak atau statis. Penggunaan \textit{observables} yang termotivasi secara fisik (yang terdapat dalam Hamiltonian) menghasilkan \textit{information gain} yang jauh lebih tinggi dibandingkan \textit{Pauli operators} acak, sehingga secara efektif menghindari masalah supresi eksponensial. Meskipun metode ini terbukti \textit{robust} pada perangkat keras IBM Kawasaki yang \textit{noisy} dibandingkan dengan \textit{baseline} variasional, hasilnya menyoroti bahwa \textit{noise} perangkat keras tetap menurunkan resolusi \textit{expectation values}, sehingga memerlukan teknik mitigasi kesalahan seperti \textit{randomized Pauli twirling} dan \textit{zero noise extrapolation} \cite{li2025}.

\section{Implementasi pada Framework Qiskit}

\subsection{Komponen Ekosistem Qiskit}
Qiskit merupakan \textit{software development kit open-source} untuk \textit{quantum information science} yang dikembangkan dengan arsitektur berbasis sirkuit. Kerangka kerja ini bertumpu pada tiga komponen utama, yaitu \textit{circuits}, \textit{pass managers}, dan \textit{primitives} \cite{sahin2025QiskitML}. Pass managers digunakan untuk mengoptimasi sirkuit kuantum, sedangkan primitives menyediakan antarmuka standar untuk mengevaluasi sirkuit pada perangkat kuantum, sehingga esensial dalam melatih model QNN \cite{sutor2019DWQ}.

Dalam komputasi kuantum, terdapat dua \textit{primitives} utama untuk menangkap keluaran sirkuit kuantum, yakni \textit{sampling output bitstrings} dan estimasi \textit{observable expectation values}. \textit{Primitives} diimplementasikan sebagai \textit{sampler} dan \textit{estimator}, dengan \textit{EstimatorQNN} memanfaatkan \textit{BaseEstimator} primitive dari Qiskit untuk mengintegrasikan sirkuit kuantum berparameter dengan \textit{observables} mekanika kuantum \cite{javadi2024Qiskit}. Dalam konteks prediksi GFD, estimator digunakan untuk mengukur nilai ekspektasi operator yang kemudian menjadi keluaran prediksi seperti probabilitas GFD tinggi \cite{munasinghe2024}.

Qiskit dirancang sebagai kerangka kerja ringan yang dapat diintegrasikan ke dalam lingkungan \textit{runtime} yang mengkolokasikan prosesor kuantum dengan prosesor klasik umum, sehingga ribuan sirkuit dapat dihasilkan dan dievaluasi secara dinamis untuk memperoleh solusi akhir. Contoh lingkungan runtime ini adalah Qiskit \textit{Runtime} yang diimplementasikan pada komputer kuantum IBM, yang memungkinkan algoritma \textit{hybrid} seperti QNN/VQR dioptimalkan secara efisien melalui interaksi intensif antara \textit{optimizer} klasik dan prosesor kuantum yang mengeksekusi sirkuit dan mengukur \textit{primitives} \cite{javadi2024Qiskit}.

\section{Tinjauan Penelitian Terdahulu dan Gap Analisis}

\subsection{Review Penelitian Terdahulu (QML in Meteorology)}
Penelitian QML dalam meteorologi dan \textit{environmental forecasting} telah menunjukkan potensi signifikan, terutama dalam menangani data deret waktu dan kompleksitas nonlinear yang tinggi \cite{pandey2025QIML}. Sebagian besar studi awal QML berfokus pada prediksi energi (seperti kecepatan angin dan iradiasi matahari) serta klasifikasi iklim, karena data tersebut memiliki karakteristik \textit{time-series} yang mirip dengan data GFD \cite{silva2025}.

Sebuah studi membandingkan berbagai pendekatan untuk memprediksi suhu permukaan global termasuk metode \textit{classical machine learning} seperti ARMA, ARIMA, dan SARIMA, jaringan saraf seperti LSTM, CNN-LSTM, dan ConvLSTM, serta teknik QML seperti QNN, VQR, dan QSVR. Hasilnya menunjukkan bahwa QSVR menjadi model unggulan untuk \textit{time-series forecasting} karena kemampuannya memanfaatkan \textit{quantum kernels} dalam menangkap pola nonlinear pada data iklim. Hal ini memvalidasi premis bahwa teknik QML berpotensi mengungguli model DL klasik untuk tugas regresi iklim yang kompleks, termasuk prediksi GFD \cite{pandey2025QIML}.

Penelitian lain yang mengimplementasikan QNN menggunakan \textit{Qiskit} untuk meramalkan \textit{global horizontal irradiance} (GHI) menunjukkan bahwa QNN mampu memberikan hasil yang kompetitif untuk horizon peramalan 5 hingga 120 menit dan bahkan mengungguli pendekatan klasik seperti SVR dan GMDH pada horizon 180 menit. Hal ini menunjukkan kapasitas QNN dalam mengidentifikasi dan mengekstrak informasi spatiotemporal dari data, serta potensi keunggulannya pada horizon peramalan yang lebih panjang, yang relevan untuk sistem peringatan dini bencana terkait GFD tinggi \cite{oliveira2024QNN}.

Pada prediksi kecepatan angin, yang juga dipandang sebagai fenomena atmosfer yang sangat \textit{chaotic}, QNN menunjukkan potensi untuk mengungguli RNN klasik dalam hal akurasi dan kemampuan beradaptasi terhadap perubahan data yang mendadak. Kemampuan QNN untuk lebih tahan terhadap data yang sangat bervariasi dan \textit{noisy} ini sangat relevan untuk prediksi GFD di wilayah tropis yang dikenal memiliki variabilitas tinggi \cite{silva2025}.

Studi lain mengevaluasi berbagai classifier termasuk SVM klasik, QSVC, dan VQC pada simulator kuantum IBM dan komputer kuantum IBM 127-\textit{qubit} menggunakan data iklim dan cuaca Earth Observation dari NASA bersama pustaka Qiskit ML. Hasilnya menunjukkan bahwa dua model kuantum (QSVC dan VQC) mampu memprediksi label kelas dengan kualitas yang wajar hanya dengan menggunakan dua \textit{qubit}, yang menegaskan kelayakan teknis penerapan QML berbasis Qiskit dengan data satelit untuk aplikasi iklim dan cuaca \cite{munasinghe2024}.

Dalam konteks cuaca ekstrem, studi lain melaporkan bahwa model kuantum mampu meningkatkan akurasi prediksi hingga 92\% untuk badai tropis dan sekaligus mempercepat waktu komputasi dari 48 jam menjadi 5 jam dibandingkan model konvensional. Temuan ini menguatkan hipotesis bahwa QNN memiliki keunggulan komputasi yang relevan untuk memprediksi fenomena cuaca ekstrem, termasuk GFD tinggi \cite{rith2025}.

\subsection{Identifikasi Gap Penelitian}
Dari perspektif geografis dan standar rekayasa, standar proteksi petir internasional seperti IEC, NFPA, IEEE, JIS, dan NEMA saat ini dikembangkan terutama berdasarkan karakteristik petir subtropis. Hal ini menimbulkan ketidaksesuaian ketika standar tersebut diterapkan di wilayah tropis, sehingga mendorong Indonesia untuk mengembangkan bidang penelitian dan lokasi uji seperti Stasiun Penelitian Petir Gunung Tangkuban Perahu (SPP-TP) dan lokasi di Bogor (SPP-Bogor) untuk mengkaji dan menginovasi sistem proteksi petir tropis \cite{denov2025RNFPA}.

Sebagaimana telah dijelaskan, terdapat disparitas signifikan antara perilaku petir di iklim tropis dan subtropis, dan investigasi selama delapan tahun di SPP-TP mengungkap berbagai atribut unik petir tropis \cite{denov2023}. Perbedaan ini menimbulkan kebutuhan akan model prediksi yang adaptif secara regional, yang tidak dapat dipenuhi oleh model QML yang dikembangkan hanya berdasarkan data iklim Amerika Utara atau Eropa yang umumnya beriklim subtropis atau lintang menengah \cite{denov2025RNFPA}.

Dari perspektif metodologis, aplikasi ML yang memanfaatkan data EO untuk memprediksi fenomena iklim memang sudah ada, namun kemampuan model-model tersebut untuk beradaptasi dari satu wilayah ke wilayah lain masih belum banyak dieksplorasi. Kesenjangan ini memperkuat urgensi penelitian yang tidak hanya mengembangkan QNN, tetapi juga menguji kemampuan generalisasi dan adaptasinya terhadap dua rezim iklim berbeda (tropis versus subtropis) yang memiliki karakteristik lightning flash yang berlainan \cite{munasinghe2024}.

Selain kesenjangan geografis, terdapat pula kesenjangan teknis terkait hardware kuantum yang masih dalam tahap pengembangan, sehingga membatasi aplikasi praktis dalam skala besar. Studi tersebut menekankan bahwa penelitian lanjutan diperlukan untuk mengembangkan sistem kuantum yang lebih stabil dan andal agar mampu memproses data cuaca yang lebih kompleks. Tesis ini diharapkan berkontribusi dengan menunjukkan bagaimana QNN dapat mencapai keunggulan komputasi dalam memodelkan interaksi variabel GFD tropis meskipun menggunakan sirkuit kuantum dangkal yang sesuai dengan keterbatasan perangkat NISQ dan ekosistem \textit{qiskit} saat ini \cite{rith2025}.