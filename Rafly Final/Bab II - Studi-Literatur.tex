% ==========================================
% BAB II STUDI LITERATUR
% ==========================================
\chapter{STUDI LITERATUR}
\label{chap:studi-literatur}
\section{Meteorologi Petir dan Ground Flash Density (GFD)}

\subsection{Konsep Dasar dan Metrik GFD}
Pengukuran dan pemodelan Ground Flash Density (GFD) sangat bergantung pada ketersediaan data petir yang andal dari pengamatan satelit. Sensor satelit seperti Lightning Imaging Sensor (LIS) dan Optical Transient Detector (OTD) mencatat petir sebagai “events” pada array pencitraan CCD, yang merepresentasikan peristiwa optik mentah akibat kilatan cahaya petir. [A Global LIS OTD Climatology of Lightning Flash Extent Density.pdf] Untuk mengidentifikasi sambaran petir yang sesungguhnya, data events ini harus melalui tahap pemrosesan tambahan berupa filtering dan clustering, sehingga diperoleh struktur data bertingkat berupa “groups”, “flashes”, dan “areas”. [A Global LIS OTD Climatology of Lightning Flash Extent Density.pdf]

Dalam pemrosesan standar LIS/OTD, “groups” didefinisikan sebagai klaster iluminasi puncak awan dari pulsa-pulsa cahaya individu, “flashes” didefinisikan sebagai rangkaian pulsa yang menggambarkan satu kilat yang utuh, sedangkan “areas” merepresentasikan klaster aktivitas petir yang menggambarkan snapshot badai. [A Global LIS OTD Climatology of Lightning Flash Extent Density.pdf] Dalam konteks GFD, penting untuk membedakan antara events, groups, dan flashes karena GFD didefinisikan sebagai jumlah sambaran petir ke darat (ground flashes) per satuan luas dan waktu (misalnya per km² per tahun), sehingga skema klasifikasi ini menjadi dasar untuk menghitung densitas sambaran yang menjadi target prediksi. [A Global LIS OTD Climatology of Lightning Flash Extent Density.pdf] LIS/OTD berfungsi sebagai salah satu sumber ground truth global utama yang relevan untuk memvalidasi model-model prediksi GFD, termasuk model berbasis komputasi kuantum yang dikembangkan pada tesis ini. [A Global LIS OTD Climatology of Lightning Flash Extent Density.pdf]

\subsection{Faktor Meteorologis Pembentuk Petir}
Pembentukan petir sangat dipengaruhi oleh kondisi termodinamika dan dinamika atmosfer yang dapat direpresentasikan melalui variabel-variabel meteorologis skala besar. Pada sebuah studi parameterisasi petir berbasis machine learning, himpunan data pelatihan dirangkum dalam suatu tabel dengan variabel input yang mencakup presipitasi (P) dan berbagai variabel medan atmosfer, seperti convective available potential energy (CAPE), ketinggian geopotensial 500 hPa (Z500), ketinggian geopotensial 1000 hPa (Z1000), ketebalan antara level 300–700 hPa (Z300–700), suhu pada 2 m (T2M), kecepatan vertikal pada 500 hPa, serta suhu titik embun pada 2 m (2-m dewpoint temperature). [Machine Learning–Based Lightning Parameterizations for the CONUS.pdf]

CAPE pada studi tersebut digunakan sebagai indikator utama ketidakstabilan atmosfer yang mendukung konveksi kuat, sedangkan presipitasi (P) dan suhu titik embun 2 m secara langsung terkait dengan ketersediaan uap air dan kelembaban yang menjadi prasyarat pembentukan awan konvektif dan petir. [Machine Learning–Based Lightning Parameterizations for the CONUS.pdf] Variabel-variabel atmosfer skala besar ini terbukti penting dalam parameterisasi petir dan, dalam konteks tesis ini, dapat difungsikan sebagai fitur input klasik untuk model Quantum Neural Network (QNN) yang dikembangkan untuk memprediksi GFD. [Machine Learning–Based Lightning Parameterizations for the CONUS.pdf]

\subsection{Disparitas Karakteristik: Tropis vs. Subtropis}
Indonesia digambarkan sebagai negara tropis yang terletak di sekitar ekuator dan dikelilingi oleh lautan, sehingga mengalami cuaca panas sepanjang tahun. [A Method to Obtain Lightning Peak Current in Indonesia.pdf] Karakter maritim dengan banyak pulau ini menghasilkan aerosol dalam jumlah besar, terutama dari garam laut, sedangkan iklim yang lembab dengan hutan, sungai, dan danau yang melimpah menjadi faktor utama pembentukan awan kumulonimbus (CB) sebagai awan penghasil petir. [A Method to Obtain Lightning Peak Current in Indonesia.pdf] Kondisi-kondisi ini menjadikan wilayah tropis maritim seperti Indonesia sangat kondusif terhadap aktivitas petir yang intensif. [A Method to Obtain Lightning Peak Current in Indonesia.pdf]

Studi lain menegaskan bahwa terdapat disparitas yang signifikan antara perilaku petir di iklim tropis dan subtropis. [Review of NFPA 780 Standard Compliance for Improved.pdf] Investigasi ekstensif selama delapan tahun di Stasiun Penelitian Petir Gunung Tangkuban Perahu menemukan berbagai karakteristik unik petir tropis, menandakan bahwa karakteristik kilat di wilayah tropis tidak dapat secara langsung disamakan dengan petir di wilayah subtropis. [A Method to Obtain Lightning Peak Current in Indonesia.pdf] Perbedaan ini menjustifikasi perlunya model prediksi yang secara eksplisit mempertimbangkan perbedaan iklim tropis dan subtropis, terutama ketika standar proteksi petir internasional yang dominan berbasis pada karakteristik subtropis hendak diterapkan di wilayah tropis. [Review of NFPA 780 Standard Compliance for Improved.pdf]

\section{Pendekatan Machine Learning Klasik}

\subsection{Algoritma Deep Learning (CNN/LSTM) untuk Nowcasting}
Dalam ranah prediksi petir berbasis komputasi klasik, deep learning telah mencapai tingkat kompleksitas arsitektur yang tinggi. Salah satu contoh adalah model LightningNet yang menggunakan arsitektur encoder–decoder dengan 20 lapisan konvolusional tiga dimensi, lapisan pooling dan upsampling, lapisan normalisasi, serta softmax classifier. [A Deep Learning Network for Cloud-to-Ground Lightning.pdf] Arsitektur ini dirancang untuk memproses data multi-sumber dalam format spasial atau citra, seperti data satelit, radar, dan lokasi petir, sehingga mampu melakukan nowcasting petir cloud-to-ground (CG) dengan resolusi yang tinggi. [A Deep Learning Network for Cloud-to-Ground Lightning.pdf] Model-model seperti LightningNet, LightNet, dan LightNet+ menunjukkan bahwa prediksi petir modern menuntut pemrosesan informasi spasial yang sangat kompleks, yang menjadi tolok ukur kinerja bagi model QNN yang akan dikembangkan. [A Deep Learning Network for Cloud-to-Ground Lightning.pdf]

Di sisi lain, recurrent neural networks (RNN) dirancang khusus untuk memproses data urutan (sequence data) dengan memprosesnya secara iteratif dalam arah sekuensial. [A Survey of Deep Learning-Based Lightning Prediction.pdf] Dalam konteks prediksi petir, RNN sering direalisasikan sebagai long short-term memory (LSTM) yang dikembangkan untuk mengatasi masalah long-term dependencies yang sulit ditangani oleh RNN standar. [A Survey of Deep Learning-Based Lightning Prediction.pdf] Prediksi GFD yang memanfaatkan deret waktu variabel-variabel meteorologis seperti CAPE dan kelembaban dari waktu ke waktu sangat relevan dengan kemampuan LSTM dalam menangkap ketergantungan temporal jangka panjang. [A Survey of Deep Learning-Based Lightning Prediction.pdf] Kombinasi CNN dan LSTM menghasilkan arsitektur ConvLSTM yang lazim digunakan untuk prediksi spatiotemporal dalam meteorologi, dan fakta ini menandakan bahwa QNN yang dikembangkan dalam penelitian ini juga harus mampu menangani dependensi temporal secara efektif. [A Survey of Deep Learning-Based Lightning Prediction.pdf]

\subsection{Keterbatasan Komputasi Klasik}
Penggunaan Earth Observation (EO) dan data cuaca dalam jumlah besar untuk prediksi cuaca ekstrem menyebabkan kompleksitas waktu komputasi yang sangat tinggi pada pendekatan komputasi klasik. [Assessment of Quantum ML Applicability for Climate Actions.pdf] Studi deep learning untuk prediksi petir, seperti LightningNet, menunjukkan bahwa model tersebut memiliki parameter yang dapat dilatih dalam jumlah sangat besar, misalnya hingga 29.128.577 parameter, sehingga pelatihan dan inferensi menjadi sangat mahal secara komputasi. [A Deep Learning Network for Cloud-to-Ground Lightning.pdf] Volume data yang besar dikombinasikan dengan kebutuhan akurasi tinggi membatasi penerapan model klasik skala penuh, dan hal ini membuka peluang bagi quantum machine learning (QML) yang menjanjikan pemrosesan data besar secara paralel dengan kecepatan dan presisi yang lebih baik. [Assessment of Quantum ML Applicability for Climate Actions.pdf]

Selain itu, atmosfer digambarkan sebagai sistem dinamis dan kacau (dynamic and chaotic system), di mana variasi kecil pada kondisi awal dapat menyebabkan perubahan signifikan pada kondisi di masa depan. [Exploring Quantum Machine Learning for Weather Forecasting.pdf] Sifat chaotic dan kompleksitas non-linear ini menyulitkan model klasik untuk memberikan prediksi yang akurat, terutama pada rentang waktu lebih panjang atau untuk peristiwa ekstrem seperti GFD tinggi. [Exploring Quantum Machine Learning for Weather Forecasting.pdf] Meskipun CNN dan LSTM mampu memodelkan sistem non-linear yang kompleks, generalisasi terhadap data yang tidak seimbang atau jarang terjadi, seperti sambaran petir ekstrem, tetap menjadi tantangan nyata bagi deep learning klasik. [Exploring Quantum Machine Learning for Weather Forecasting.pdf]

Studi lain menunjukkan bahwa ketika badai petir disimulasikan pada skala yang lebih kecil dengan resolusi spasial dan temporal yang lebih halus, prediktabilitas kejadiannya justru berkurang. [A machine-learning approach to thunderstorm forecasting.pdf] Akurasi prediksi cenderung menurun seiring dengan peningkatan resolusi karena ketidakpastian prakiraan numerik (NWP) yang meningkat, sehingga terdapat batas fisik pada prediktabilitas sistem atmosfer. [A machine-learning approach to thunderstorm forecasting.pdf] Dalam konteks ini, QML diharapkan dapat mengimbangi peningkatan ketidakpastian melalui keunggulan komputasi dan representasi ruang fitur berdimensi tinggi. [Assessment of Quantum ML Applicability for Climate Actions.pdf]

\section{Landasan Teori Komputasi Kuantum}

\subsection{Qubit dan Ruang Hilbert}
Elemen dasar dari komputer kuantum adalah qubit, yang memiliki definisi matematis yang tepat sebagai unsur dalam suatu Ruang Hilbert kompleks. [Quantum Computers for Weather and Climate Prediction The Good, the Bad and the Noisy.pdf] Ruang Hilbert tersebut merupakan ruang vektor berdimensi kompleks yang dilengkapi dengan inner product, di mana vektor-vektor di dalamnya memiliki amplitudo kompleks. [Quantum Computers for Weather and Climate Prediction The Good, the Bad and the Noisy.pdf] Dalam konteks komputasi kuantum, qubit direpresentasikan sebagai vektor dalam Ruang Hilbert kompleks dua dimensi, dinotasikan sebagai 
RUMUS
. [Quantum Computers for Weather and Climate Prediction The Good, the Bad and the Noisy.pdf] Qubit dijelaskan sebagai kombinasi linear dari basis komputasional, meskipun bentuk eksplisit basis tersebut tidak disajikan secara lengkap dalam kutipan. [Quantum Computers for Weather and Climate Prediction The Good, the Bad and the Noisy.pdf] [KURANG INFORMASI MENGENAI bentuk eksplisit representasi vektor qubit dalam basis komputasional]

Keunggulan utama komputer kuantum berasal dari kemampuan untuk “meng-entangle” qubit, sehingga membentuk keadaan multi-qubit melalui tensor product dari vektor-vektor di Ruang Hilbert. [Quantum Computers for Weather and Climate Prediction The Good, the Bad and the Noisy.pdf] Keadaan n-qubit dapat dituliskan sebagai kombinasi linear dari 
RUMUS
basis state, sehingga secara efektif diasosiasikan dengan
RUMUS
bilangan kompleks yang menggambarkan amplitudo setiap basis. [Quantum Computers for Weather and Climate Prediction The Good, the Bad and the Noisy.pdf] Peningkatan dimensi Ruang Hilbert yang berskala eksponensial terhadap jumlah qubit ini memiliki konsekuensi yang penting, karena memungkinkan representasi informasi dalam jumlah sangat besar secara intrinsik paralel. [Quantum Computers for Weather and Climate Prediction The Good, the Bad and the Noisy.pdf] Entanglement dapat diciptakan melalui operasi seperti Controlled-NOT (CNOT), sehingga qubit-kuibit secara kolektif dapat merepresentasikan struktur data kompleks yang relevan, misalnya data cuaca. [Quantum Computers for Weather and Climate Prediction The Good, the Bad and the Noisy.pdf]

Proses pengukuran dalam mekanika kuantum diatur oleh Aturan Born, yang menyatakan bahwa probabilitas mengamati suatu keluaran tertentu saat mengukur keadaan kuantum diberikan oleh ekspresi matematis tertentu yang tidak dicantumkan secara lengkap dalam kutipan. [Quantummachinelearning.pdf] Pengukuran merupakan proses non-reversibel yang menyebabkan keadaan superposisi kolaps menjadi salah satu status dasar. [Quantummachinelearning.pdf] [KURANG INFORMASI MENGENAI bentuk matematis lengkap Aturan Born yang digunakan]


\subsection{Quantum Data Encoding/Feature Map}
Feature map merupakan komponen fundamental dalam sirkuit QNN karena berfungsi memetakan input klasik ke dalam ruang kuantum yang didefinisikan di dalam Ruang Hilbert yang sangat kompleks. [Application of Quantum Neural Network.pdf] Tahap ini dianggap sebagai bagian paling kritis dalam desain QML karena mentransformasikan struktur data secara mendalam dan non-linear dengan cara yang tidak dapat dicapai oleh metode klasik. [Application of Quantum Neural Network.pdf] Dengan memproyeksikan data GFD tropis dan subtropis ke ruang fitur kuantum berdimensi tinggi, relasi non-linear antara variabel meteorologis dapat dieksplorasi dan diekstraksi secara lebih efisien dibandingkan pendekatan kernel klasik. [Application of Quantum Neural Network.pdf]

Salah satu metode encoding yang umum dan praktis untuk perangkat NISQ adalah angle embedding atau angle encoding. [Quantum machine learning early.pdf] Metode ini meng-encode vektor d-dimensi 
RUMUS
ke dalam sudut rotasi gerbang qubit tunggal, sehingga untuk 
RUMUS
n≈d qubit diperoleh operator 
RUMUS
 dengan 
RUMUS
P∈{X,Y,Z}. [Quantum machine learning early.pdf] Pada konteks GFD, nilai fitur seperti CAPE atau kelembaban dapat dimasukkan langsung sebagai parameter sudut rotasi pada gerbang Pauli 
RUMUS
. [Application of Quantum Neural Network.pdf] Hal ini mengimplikasikan bahwa setiap variabel input atmosfer memerlukan setidaknya satu qubit tersendiri, sehingga jumlah fitur input dibatasi oleh jumlah qubit yang tersedia pada perangkat NISQ. [Application of Quantum Neural Network.pdf]

Metode lain adalah amplitude embedding, yang meng-encode vektor fitur berdimensi N yang ternormalisasi ke dalam amplitudo keadaan n-qubit dengan syarat 
RUMUS
. [Quantum machine learning early.pdf] Pendekatan ini sangat efisien dari sisi jumlah qubit karena mampu memetakan hingga 
RUMUS
fitur ke dalam n qubit, tetapi implementasi fisiknya pada perangkat NISQ dianggap menantang. [Quantum machine learning early.pdf] Walaupun menawarkan efisiensi qubit yang superior, keterbatasan implementasi ini menjadikan amplitude encoding kurang praktis untuk model GFD pada era NISQ dibandingkan angle encoding atau pendekatan encoding lain di ruang laten yang lebih sesuai dengan hardware saat ini. [Quantum machine learning early.pdf]

ZZ Feature Map merupakan contoh feature map yang lebih kompleks dan dapat diimplementasikan menggunakan Qiskit. [Application of Quantum Neural Network.pdf] Map ini tidak hanya meng-encode data, tetapi juga secara eksplisit memanfaatkan gerbang Hadamard dan CNOT untuk menciptakan entanglement dan mengkodekan interaksi antar fitur 
RUMUS
melalui fase kuantum. [Application of Quantum Neural Network.pdf] Pendekatan ini sangat relevan untuk GFD karena interaksi non-linear antara variabel meteorologis, seperti CAPE dan kelembaban, berperan penting terhadap akurasi prediksi. [Application of Quantum Neural Network.pdf] [KURANG INFORMASI MENGENAI bentuk lengkap sirkuit ZZ feature map]\end{enumerate}

\section{Arsitektur Model Kuantum}

\subsection{Variational Quantum Circuits (VQC)}
Variational Quantum Circuits (VQC), yang juga dikenal sebagai Variational Quantum Algorithms (VQA) atau Quantum Neural Networks (QNN), merupakan strategi utama untuk mencapai keunggulan komputasi kuantum menggunakan perangkat NISQ (Noisy Intermediate-Scale Quantum). [Variational quantum algorithms.pdf] VQA memanfaatkan classical optimizer untuk melatih sirkuit kuantum berparameter (parameterized quantum circuit), sehingga membentuk kerangka kerja hybrid kuantum-klasik. [Variational quantum algorithms.pdf]

Secara arsitektural, model QNN terdiri dari tiga komponen dasar, yaitu Feature Map, Ansatz, dan Measurement, yang bersama-sama membentuk sirkuit kuantum tempat data diproses. [Application of Quantum Neural Network.pdf] Kerangka EstimatorQNN pada Qiskit memanfaatkan sirkuit kuantum campuran yang terdiri dari dua komponen terpisah: feature map dan ansatz. [Application of Quantum Neural Network.pdf] Dalam konteks tesis ini, Feature Map digunakan untuk meng-encode data meteorologis, Ansatz untuk memproses dan melatih parameter kuantum, dan Measurement untuk memperoleh keluaran prediksi GFD. [Application of Quantum Neural Network.pdf]

Ansatz dalam QNN merujuk pada tebakan awal untuk suatu keadaan kuantum, yang berperan sebagai titik awal perhitungan. [Application of Quantum Neural Network.pdf] Dalam analogi dengan machine learning klasik, ansatz berperan seperti bobot pelatihan, di mana sirkuit kuantum berparameter ini dioptimalkan untuk meminimalkan fungsi biaya tertentu. [Application of Quantum Neural Network.pdf] Parameter dalam ansatz (misalnya sudut rotasi θ) dilatih pada setiap iterasi menggunakan algoritma optimasi klasik yang dijalankan pada komputer klasik, sehingga membentuk hybrid loop VQC. [Application of Quantum Neural Network.pdf]

Beberapa jenis ansatz yang umum digunakan pada framework Qiskit antara lain TwoLocal ansatz dan Ry ansatz. [Application of Quantum Neural Network.pdf] TwoLocal ansatz terdiri atas lapisan-lapisan rotasi dan entanglement dengan gerbang berparameter, di mana gerbang rotasi diterapkan pada setiap qubit secara individual, sedangkan entanglement diterapkan pada dua qubit sesuai strategi entanglement yang diinginkan. [Application of Quantum Neural Network.pdf] Ry ansatz memanfaatkan gerbang Pauli-Y berparameter sebagai blok rotasi, sedangkan QAOA (Quantum Approximate Optimization Algorithm) digunakan sebagai ansatz yang merupakan kasus khusus dari VQE (Variational Quantum Eigensolver). [Application of Quantum Neural Network.pdf] Ansatz-ansatz ini relevan untuk merancang arsitektur VQC dalam tugas prediksi GFD. [Application of Quantum Neural Network.pdf]

\subsection{QNN vs. QSVM}
Prediksi GFD merupakan tugas regresi karena GFD berupa nilai kontinu, sehingga perbandingan antara QNN/VQR (Variational Quantum Regressor) dan QSVR (Quantum Support Vector Regressor) menjadi penting. [A comparative analysis of classical machine learning models with quantum-inspired models....pdf] Sebuah studi peramalan deret waktu suhu permukaan global yang membandingkan berbagai pendekatan, termasuk ARMA, ARIMA, SARIMA, LSTM, CNN-LSTM, ConvLSTM, dan teknik QML seperti QNN, VQR, dan QSVR, menemukan bahwa QSVR menjadi model yang menonjol untuk time-series forecasting. [A comparative analysis of classical machine learning models with quantum-inspired models....pdf] Keunggulan QSVR tersebut dikaitkan dengan kemampuannya memanfaatkan quantum kernels untuk menangkap pola non-linear dalam data iklim. [A comparative analysis of classical machine learning models with quantum-inspired models....pdf]

Di sisi lain, QNN dipandang sebagai frontier baru di bidang komputasi kuantum yang memutuskan diri dari struktur jaringan saraf tradisional sambil mempertahankan prinsip-prinsip fundamentalnya. [The power of quantum neural networks.pdf] Tidak seperti jaringan saraf klasik, QNN tidak memiliki neuron kuantum diskrit, dan keluaran model diekstraksi melalui fungsi post-processing klasik yang diterapkan pada hasil pengukuran (measurement outcome) dari sirkuit kuantum. [The power of quantum neural networks.pdf] Dalam konteks regresi GFD, nilai prediksi kontinu dapat diperoleh dari nilai ekspektasi suatu operator (observable) pada qubit keluaran, kemudian diolah lebih lanjut dengan fungsi klasik. [Solving nonlinear differential equations....pdf]

Dari sisi kompleksitas komputasi, pelatihan QSVR memerlukan perhitungan jarak pasangan (pair-wise distances) antara seluruh sampel pelatihan dalam matriks Gram berukuran 
RUMUS
M×M, sehingga kompleksitas waktunya paling tidak 
RUMUS
) terhadap jumlah sampel pelatihan M. [Supervised quantum machine learning models....pdf] Sebaliknya, pelatihan jaringan saraf, termasuk QNN/VQR yang dioptimalkan dengan pendekatan gradient-based, memiliki kompleksitas waktu 
RUMUS
O(M) yang bergantung secara linear pada jumlah sampel pelatihan. [Supervised quantum machine learning models....pdf] [KURANG INFORMASI MENGENAI orde kompleksitas waktu pelatihan QNN secara eksplisit di luar pernyataan umum 
RUMUS

Studi lain menunjukkan bahwa QNN mampu mencapai effective dimension yang secara signifikan lebih baik dibandingkan jaringan saraf klasik yang sebanding, yang dikaitkan dengan lanskap optimasi yang lebih menguntungkan dan spektrum Fisher information yang lebih merata. [The power of quantum neural networks.pdf] Daya ekspresif (expressibility) yang lebih tinggi ini menjadikan QNN kandidat potensial untuk memodelkan dinamika GFD yang sangat non-linear dan chaotic, terutama ketika kompleksitas data menuntut ansatz yang lebih dalam dengan banyak lapisan. [The power of quantum neural networks.pdf]

\section{Implementasi pada Framework Qiskit}

\subsection{Komponen Ekosistem Qiskit}
Qiskit merupakan software development kit open-source untuk quantum information science yang dikembangkan dengan arsitektur berbasis sirkuit. [Qiskit: An Open-source Framework for Quantum Computing.pdf] Kerangka kerja ini bertumpu pada tiga komponen utama, yaitu circuits, pass managers, dan primitives. [Qiskit: An Open-source Framework for Quantum Computing.pdf] Pass managers digunakan untuk mengoptimasi sirkuit kuantum, sedangkan primitives menyediakan antarmuka standar untuk mengevaluasi sirkuit pada perangkat kuantum, sehingga esensial dalam melatih model QNN. [Dancing with Qubits.pdf]

Dalam komputasi kuantum, terdapat dua primitives utama untuk menangkap keluaran sirkuit kuantum, yakni sampling output bitstrings dan estimasi observable expectation values. [Quantum computing with Qiskit.pdf] Primitives diimplementasikan sebagai sampler dan estimator, di mana EstimatorQNN memanfaatkan BaseEstimator primitive dari Qiskit untuk mengintegrasikan sirkuit kuantum berparameter dengan observables mekanika kuantum. [Quantum computing with Qiskit.pdf] Dalam konteks prediksi GFD, estimator digunakan untuk mengukur nilai ekspektasi operator seperti Pauli 
RUMUS
X, yang kemudian menjadi keluaran prediksi seperti probabilitas GFD tinggi. [Assessment of Quantum ML Applicability for Climate Actions...pdf]

Qiskit dirancang sebagai kerangka kerja ringan yang dapat diintegrasikan ke dalam lingkungan runtime yang mengkolokasikan prosesor kuantum dengan prosesor klasik umum, sehingga ribuan sirkuit dapat dihasilkan dan dievaluasi secara dinamis untuk memperoleh solusi akhir. [Quantum computing with Qiskit.pdf] Contoh lingkungan runtime ini adalah Qiskit Runtime yang diimplementasikan pada komputer kuantum IBM, yang memungkinkan algoritma hybrid seperti QNN/VQR dioptimalkan secara efisien melalui interaksi intensif antara optimizer klasik (misalnya COBYLA atau Adam) dan prosesor kuantum yang mengeksekusi sirkuit dan mengukur primitives. [Quantum machine learning early.pdf]

\subsection{Penanganan Noise pada Era NISQ}
Perangkat kuantum saat ini beroperasi pada era NISQ (noisy intermediate-scale quantum), yang dicirikan oleh jumlah qubit yang terbatas (puluhan hingga beberapa ratus) dan kerentanan terhadap error akibat noise lingkungan dan ketidaksempurnaan kontrol. [Quantum machine learning early.pdf] Qubit pada era ini memiliki waktu koherensi yang pendek, sehingga sirkuit kuantum harus dirancang dengan kedalaman yang dangkal (shallow depth) agar hasil pengukuran tetap valid. [Variational quantum algorithms.pdf] Kondisi ini menjadi motivasi utama penggunaan strategi VQA/VQC dalam tesis ini karena algoritma variational dirancang khusus untuk beroperasi pada perangkat NISQ. [Application of Quantum Neural Network.pdf]

Dalam jangka panjang, tujuan utama bidang komputasi kuantum adalah mengimplementasikan Quantum Error Correction (QEC), namun QEC merupakan tantangan rekayasa kuantum tingkat lanjut yang membutuhkan ribuan qubit fisik untuk mewujudkan satu qubit logis yang bebas noise, sehingga belum praktis pada perangkat sekarang. [Dancing with Qubits.pdf] Untuk era NISQ, pendekatan yang lebih realistis adalah error mitigation (EM) berbasis perangkat lunak, yang digunakan untuk mengestimasi hasil bebas noise dari keluaran yang bising. [Quantum machine learning early.pdf] Metode-metode EM dapat digabungkan dengan VQA untuk meningkatkan akurasi hasil, misalnya ketika model dijalankan pada perangkat nyata seperti IBM 127-qubit Eagle. [Assessment of Quantum ML Applicability for Climate Actions...pdf]

Salah satu teknik EM yang menonjol adalah zero-noise extrapolation (ZNE). [Variational quantum algorithms.pdf] Teknik ini memanfaatkan fakta bahwa meskipun laju error tidak dapat dikurangi, dalam banyak kasus noise dapat sengaja ditingkatkan; dengan menjalankan sirkuit pada beberapa tingkat noise yang berbeda dan mengumpulkan hasil pengukuran, kemudian dilakukan proses ekstrapolasi untuk mengestimasi hasil pada batas noise nol. [Quantum computing with Qiskit.pdf] Pendekatan ZNE ini melibatkan post-processing klasik terhadap hasil pengukuran dan digunakan untuk meningkatkan akurasi nilai ekspektasi yang dihitung oleh estimator. [Variational quantum algorithms.pdf]

\section{Tinjauan Penelitian Terdahulu dan Gap Analisis}

\subsection{Review Penelitian Terdahulu (QML in Meteorology)}
Penelitian QML dalam meteorologi dan environmental forecasting telah menunjukkan potensi signifikan, terutama dalam menangani data deret waktu dan kompleksitas non-linear yang tinggi. [A comparative analysis of classical machine learning models with quantum-inspired models....pdf] Sebagian besar studi awal QML berfokus pada prediksi energi (seperti kecepatan angin dan iradiasi matahari) serta klasifikasi iklim, karena data tersebut memiliki karakteristik time-series yang mirip dengan data GFD. [Exploring Quantum Machine Learning for Weather Forecasting.pdf]

Sebuah studi membandingkan berbagai pendekatan untuk memprediksi suhu permukaan global termasuk metode classical machine learning seperti ARMA, ARIMA, dan SARIMA, jaringan saraf seperti LSTM, CNN-LSTM, dan ConvLSTM, serta teknik QML seperti QNN, VQR, dan QSVR. [A comparative analysis of classical machine learning models with quantum-inspired models....pdf] Hasilnya menunjukkan bahwa QSVR menjadi model unggulan untuk time-series forecasting karena kemampuannya memanfaatkan quantum kernels dalam menangkap pola non-linear pada data iklim. [A comparative analysis of classical machine learning models with quantum-inspired models....pdf] Hal ini memvalidasi premis bahwa teknik QML berpotensi mengungguli model DL klasik untuk tugas regresi iklim yang kompleks, termasuk prediksi GFD. [A comparative analysis of classical machine learning models with quantum-inspired models....pdf]

Penelitian lain yang mengimplementasikan QNN menggunakan Qiskit untuk meramalkan global horizontal irradiance (GHI) menunjukkan bahwa QNN mampu memberikan hasil yang kompetitif untuk horizon peramalan 5 hingga 120 menit dan bahkan mengungguli pendekatan klasik seperti SVR dan GMDH pada horizon 180 menit. [Application of Quantum Neural Network.pdf] Hal ini menunjukkan kapasitas QNN dalam mengidentifikasi dan mengekstrak informasi spatiotemporal dari data, serta potensi keunggulannya pada horizon peramalan yang lebih panjang, yang relevan untuk sistem peringatan dini bencana terkait GFD tinggi. [Application of Quantum Neural Network.pdf]

Pada prediksi kecepatan angin, yang juga dipandang sebagai fenomena atmosfer yang sangat chaotic, QNN menunjukkan potensi untuk mengungguli Recurrent Neural Network (RNN) klasik dalam hal akurasi dan kemampuan beradaptasi terhadap perubahan data yang mendadak. [Exploring Quantum Machine Learning for Weather Forecasting.pdf] Kemampuan QNN untuk lebih tahan terhadap data yang sangat bervariasi dan noisy ini sangat relevan untuk prediksi GFD di wilayah tropis yang dikenal memiliki variabilitas tinggi. [Exploring Quantum Machine Learning for Weather Forecasting.pdf]

Studi lain mengevaluasi berbagai classifier termasuk SVM klasik, QSVC, dan VQC pada simulator kuantum IBM dan komputer kuantum IBM 127-qubit menggunakan data iklim dan cuaca Earth Observation dari NASA bersama pustaka Qiskit ML 0.7.2. [Assessment of Quantum ML Applicability for Climate Actions...pdf] Hasilnya menunjukkan bahwa dua model kuantum (QSVC dan VQC) mampu memprediksi label kelas dengan kualitas yang wajar hanya dengan menggunakan dua qubit, yang menegaskan kelayakan teknis penerapan QML berbasis Qiskit dengan data satelit untuk aplikasi iklim dan cuaca. [Assessment of Quantum ML Applicability for Climate Actions...pdf]

Dalam konteks cuaca ekstrem, studi lain melaporkan bahwa model kuantum mampu meningkatkan akurasi prediksi hingga 92 persen untuk badai tropis dan sekaligus mempercepat waktu komputasi dari 48 jam menjadi 5 jam dibandingkan model konvensional. [Quantum Computing to Forecast Extreme Weather.pdf] Temuan ini menguatkan hipotesis bahwa QNN memiliki keunggulan komputasi yang relevan untuk memprediksi fenomena cuaca ekstrem, termasuk GFD tinggi. [Quantum Computing to Forecast Extreme Weather.pdf]

\subsection{Identifikasi Gap Penelitian}
Dari perspektif geografis dan standar rekayasa, standar proteksi petir internasional seperti IEC, NFPA, IEEE, JIS, dan NEMA saat ini dikembangkan terutama berdasarkan karakteristik petir subtropis. [Review of NFPA 780 Standard Compliance for Improved.pdf] Hal ini menimbulkan ketidaksesuaian ketika standar tersebut diterapkan di wilayah tropis, sehingga mendorong Indonesia untuk mengembangkan bidang penelitian dan lokasi uji seperti Stasiun Penelitian Petir Gunung Tangkuban Perahu (SPP-TP) dan lokasi di Bogor (SPP-Bogor) untuk mengkaji dan menginovasi sistem proteksi petir tropis. [Review of NFPA 780 Standard Compliance for Improved.pdf]

Sebagaimana telah dijelaskan, terdapat disparitas signifikan antara perilaku petir di iklim tropis dan subtropis, dan investigasi selama delapan tahun di SPP-TP mengungkap berbagai atribut unik petir tropis. [A Method to Obtain Lightning Peak Current in Indonesia.pdf] Perbedaan ini menimbulkan kebutuhan akan model prediksi yang adaptif secara regional, yang tidak dapat dipenuhi oleh model QML yang dikembangkan hanya berdasarkan data iklim Amerika Utara atau Eropa yang umumnya beriklim subtropis atau lintang menengah. [Review of NFPA 780 Standard Compliance for Improved.pdf]

Dari perspektif metodologis, aplikasi ML yang memanfaatkan data EO untuk memprediksi fenomena iklim memang sudah ada, namun kemampuan model-model tersebut untuk beradaptasi dari satu wilayah ke wilayah lain masih belum banyak dieksplorasi. [Assessment of Quantum ML Applicability for Climate Actions...pdf] Kesenjangan ini memperkuat urgensi penelitian yang tidak hanya mengembangkan QNN, tetapi juga menguji kemampuan generalisasi dan adaptasinya terhadap dua rezim iklim berbeda (tropis vs. subtropis) yang memiliki karakteristik lightning flash yang berlainan. [Assessment of Quantum ML Applicability for Climate Actions...pdf]

Selain kesenjangan geografis, terdapat pula kesenjangan teknis terkait hardware kuantum yang masih dalam tahap pengembangan, sehingga membatasi aplikasi praktis dalam skala besar. [Quantum Computing to Forecast Extreme Weather.pdf] Studi tersebut menekankan bahwa penelitian lanjutan diperlukan untuk mengembangkan sistem kuantum yang lebih stabil dan andal agar mampu memproses data cuaca yang lebih kompleks. [Quantum Computing to Forecast Extreme Weather.pdf] Tesis ini diharapkan berkontribusi dengan menunjukkan bagaimana QNN dapat mencapai keunggulan komputasi dalam memodelkan interaksi variabel GFD tropis meskipun menggunakan sirkuit kuantum dangkal yang sesuai dengan keterbatasan perangkat NISQ dan ekosistem Qiskit saat ini. [Quantum Computing to Forecast Extreme Weather.pdf]