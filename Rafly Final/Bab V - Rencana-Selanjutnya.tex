% ==========================================
% BAB V RENCANA SELANJUTNYA
% ==========================================
\chapter{RENCANA SELANJUTNYA}
\label{chap:rencana-selanjutnya}
V.1 Rencana Implementasi

Tahap persiapan dan pematangan studi literatur

Menyelesaikan studi literatur dengan protokol PRISMA dan Concept Matrix untuk memfinalisasi daftar variabel meteorologis dan arsitektur QNN yang paling relevan dengan GFD.
Merumuskan lebih rinci skenario eksperimen (in-domain vs cross-domain, baseline vs QNN).

Pengumpulan dan integrasi data

Memperoleh dataset petir dari PLN (wilayah tropis Indonesia) dan data LIS/OTD atau produk NASA lain untuk wilayah subtropis/lintang menengah sesuai ruang lingkup.
Melakukan rekonsiliasi format, sistem koordinat, dan resolusi spasial-temporal; menghitung nilai GFD per grid/wilayah sesuai definisi penelitian.

Preprocessing dan pemilihan fitur

Membersihkan data dari outlier, data ganda, dan missing values, serta menyesuaikan periode waktu yang overlap antara dataset.
Memilih dan/atau membangun fitur meteorologis (CAPE, presipitasi, suhu, kelembapan, dsb.) berdasarkan literatur dan ketersediaan data, kemudian melakukan normalisasi/standarisasi.

Implementasi lingkungan pemodelan

Menyiapkan lingkungan Python, Jupyter Notebook, dan pustaka yang diperlukan (Qiskit, scikit-learn, pustaka numerik) di laptop penelitian.
Mengonfigurasi akses ke IBM Quantum (jika memungkinkan) untuk uji coba terbatas di hardware nyata, dengan tetap menjadikan simulator sebagai platform utama.

Pembangunan dan pelatihan model

Mengimplementasikan baseline klasik (mis. RF, SVR, MLP sederhana) pada dataset GFD terintegrasi sesuai skenario train/test yang direncanakan.
Mendesain dan mengimplementasikan QNN/VQC di Qiskit: memilih feature map dan ansatz, mengatur jumlah qubit, kedalaman sirkuit, optimizer, dan hyperparameter lain; kemudian melakukan training dan tuning bertahap.

Dokumentasi dan pengelolaan eksperimen

Mencatat semua konfigurasi, seed, dan hasil eksperimen dalam log/Spreadsheet untuk memudahkan analisis akhir.
Menyusun struktur folder dan notebook yang terorganisir (data, preprocessing, baseline, QNN, evaluasi) sebagai dasar penulisan BAB hasil dan pembahasan.

V.2 Desain Pengujian dan Evaluasi

Strategi pembagian data dan skenario uji

Membagi data menjadi set train, validation, dan test dengan mempertimbangkan pemisahan domain (tropis vs subtropis) dan/atau jendela waktu tertentu.
Menyusun beberapa skenario:
Skenario in-domain (train dan test dalam domain yang sama).
Skenario cross-domain (train tropis – test subtropis, dan sebaliknya).

Metrik evaluasi dan prosedur pengukuran

Menetapkan metrik utama seperti MSE, RMSE, MAE, dan R² untuk mengukur akurasi prediksi GFD.
Menggunakan metrik yang sama untuk seluruh model (klasik dan QNN) agar perbandingan adil dan transparan.

Perbandingan dengan baseline klasik

Menghasilkan tabel dan grafik perbandingan kinerja baseline (RF/SVR/MLP) vs QNN pada setiap skenario (in-domain, cross-domain).
Menganalisis kondisi di mana QNN memberikan perbaikan (atau penurunan) performa dibanding baseline, termasuk perbedaan perilaku pada GFD tinggi.

Evaluasi generalisasi tropis–subtropis

Mengkaji error distribusi per domain untuk melihat apakah model yang dilatih di satu domain tetap stabil ketika diuji di domain lain.
Menginterpretasikan hasil generalisasi dalam konteks perbedaan karakteristik petir tropis vs subtropis dan implikasinya bagi desain proteksi.

Analisis sensitivitas dan ablation study sederhana

Mencoba variasi jumlah fitur, tipe feature map, kedalaman ansatz, dan pilihan optimizer untuk menilai sensitivitas kinerja QNN terhadap desain arsitektur.
Mendokumentasikan konfigurasi yang paling stabil dan efisien sebagai rekomendasi desain QNN untuk GFD.

V.3 Analisis Risiko dan Mitigasi

Risiko kualitas dan ketersediaan data

Potensi ketidaklengkapan data (gap temporal, missing lokasi, noise) dari PLN atau NASA yang dapat menurunkan kualitas model.
Mitigasi: menerapkan prosedur cleaning dan imputasi yang ketat, menyempitkan ruang lingkup ke periode/wilayah dengan kualitas data terbaik, serta menyiapkan alternatif fokus (mis. satu domain) jika data domain lain terlalu terbatas.

Risiko ketidakseimbangan dan non-stationarity GFD

Nilai GFD tinggi mungkin jarang, sehingga model cenderung bias ke nilai rendah/normal dan gagal menangkap ekor distribusi.
Mitigasi: eksplorasi transformasi target (mis. log-GFD), penimbangan error (weighted loss), atau teknik resampling sederhana; fokus analisis khusus pada subset kejadian GFD tinggi.

Risiko keterbatasan komputasi dan konvergensi QNN

Simulator Qiskit dapat menjadi lambat untuk kombinasi qubit dan kedalaman sirkuit tertentu; pelatihan QNN berisiko mengalami barren plateaus atau tidak konvergen.
Mitigasi: memulai dengan arsitektur sangat dangkal dan sedikit qubit, meningkatkan kompleksitas secara bertahap; mencoba beberapa jenis ansatz dan optimizer; menyiapkan baseline klasik yang solid sebagai fallback jika QNN tidak stabil.

Risiko akses dan stabilitas hardware kuantum

Akses ke IBM Quantum mungkin terbatas (antrian panjang, kuota tembak), dan noise perangkat dapat menurunkan kualitas hasil.
Mitigasi: menjadikan simulator sebagai platform utama untuk seluruh analisis kuantitatif; menggunakan hardware nyata hanya sebagai demonstrasi kualitatif berskala kecil jika memungkinkan.

Risiko manajemen waktu dan lingkup skripsi

Batas waktu pengerjaan skripsi dapat terancam jika eksperimen terlalu banyak atau kompleksitas model tidak dikendalikan.
Mitigasi: menetapkan milestone sesuai fase CRISP-DM, memprioritaskan skenario eksperimen utama (mis. satu baseline + satu konfigurasi QNN inti), dan membatasi eksplorasi tambahan hanya jika waktu memungkinkan.

Risiko kesulitan implementasi teknis Qiskit/QML

Kurva belajar Qiskit dan QML relatif curam, sehingga ada risiko bottleneck pada fase coding dan debugging.
Mitigasi: memanfaatkan dokumentasi resmi Qiskit, contoh kode EstimatorQNN, dan memecah implementasi menjadi modul kecil (encoding, ansatz, training) yang diuji satu per satu sebelum digabungkan.