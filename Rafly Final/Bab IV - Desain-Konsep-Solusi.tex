% ==========================================
% BAB IV DESAIN KONSEP SOLUSI
% ==========================================
\chapter{DESAIN KONSEP SOLUSI}
\label{chap:desain-konsep-solusi}
Arsitektur umum solusi usulan (gambaran end-to-end)

Input: kumpulan data petir dan meteorologi dari PLN (wilayah tropis) dan NASA/LIS-OTD (global/subtropis) yang telah dipilih sesuai ruang lingkup penelitian.
Modul preprocessing: pembersihan data, rekonsiliasi format, agregasi spasial-temporal, perhitungan GFD, dan konstruksi fitur meteorologis.
Modul pemodelan klasik: implementasi baseline (mis. RF/SVR/MLP) untuk memprediksi GFD dengan pipeline yang serupa.
Modul QNN/VQC berbasis Qiskit: encoding data ke ruang kuantum (feature map), pemrosesan melalui ansatz parametrik, dan measurement via EstimatorQNN untuk menghasilkan prediksi GFD.
Modul evaluasi: perhitungan metrik (MSE, RMSE, MAE, R²), analisis error, dan visualisasi perbandingan tropis vs subtropis serta QNN vs model klasik.

Komponen utama sistem yang diusulkan

Sumber data:

Dataset sambaran petir lokal (mis. PLN) yang berisi lokasi dan waktu sambaran di wilayah tropis Indonesia.
Dataset satelit NASA/LIS-OTD yang menyediakan informasi flashes dan/atau GFD di wilayah subtropis dan global.

Modul pengolahan data:

Sinkronisasi resolusi spasial (grid) dan temporal antara dataset; kalkulasi GFD per grid.
Normalisasi dan scaling fitur sehingga sesuai untuk encoding kuantum (mis. rentang sudut rotasi).

Modul Qiskit (inti kuantum):

Feature map (mis. angle encoding atau ZZFeatureMap) untuk memetakan fitur meteorologis ke state qubit.
Ansatz parametrik (mis. TwoLocal/Ry) yang membentuk VQC dengan kedalaman terkendali.
Estimator primitive untuk menghitung expectation value sebagai output regresi GFD.

Modul evaluasi dan analisis:

Fungsi evaluasi terstandardisasi untuk semua model (klasik dan QNN).
Fasilitas untuk membandingkan performa di domain tropis dan subtropis serta skenario cross-domain.

Sistem saat ini (before) – karakteristik umum

Peta GFD dan parameter petir banyak bergantung pada model numerik, statistik klimatologis, atau ML klasik yang dikembangkan di wilayah subtropis/lintang menengah.
Integrasi eksplisit antara data tropis Indonesia (PLN) dan data global/subtropis (NASA/LIS-OTD) dalam satu kerangka pemodelan masih terbatas.
Belum ada pemanfaatan QNN/VQC sebagai alternatif atau pelengkap model klasik untuk tugas prediksi GFD.

Sistem usulan (after) – konsep solusi QML untuk GFD

Mengintegrasikan data tropis dan subtropis dalam satu pipeline analitik, dengan penandaan domain yang jelas untuk studi generalisasi.
Menyediakan dua jalur pemodelan: jalur klasik (baseline) dan jalur QNN/VQC di Qiskit, sehingga dapat dievaluasi berdampingan pada data dan metrik yang sama.
Menghasilkan fungsi prediktor GFD berbasis QNN yang siap diuji untuk berbagai konfigurasi (in-domain vs cross-domain, variasi fitur, variasi ansatz).
Membuka ruang untuk pengujian di perangkat kuantum nyata melalui desain sirkuit yang dangkal dan hemat qubit.

Alur data dan pemodelan (konsep block diagram)

Data mentah (PLN + NASA) → modul preprocessing & integrasi → dataset fitur siap pakai (tropis, subtropis, label GFD).
Dataset fitur → pembagian train/validation/test (dengan skenario domain-aware) →

Cabang 1: training model klasik (RF/SVR/MLP) + evaluasi baseline.
Cabang 2: encoding data ke sirkuit QNN (feature map + ansatz) → training VQC dengan optimizer klasik → evaluasi QNN.

Hasil evaluasi → modul analisis (perbandingan model, analisis error, visualisasi pola tropis vs subtropis) → rekomendasi dan insight untuk standar/proteksi petir.

Keterkaitan dengan metodologi (PRISMA, Concept Matrix, CRISP-DM)

PRISMA + Concept Matrix digunakan di awal untuk mengidentifikasi literatur terkait GFD, ML, dan QML, sehingga desain solusi merefleksikan gap yang telah dipetakan.
CRISP-DM memandu alur: Business Understanding (urgensi GFD tropis), Data Understanding & Preparation (integrasi PLN + NASA), Modeling (baseline klasik & QNN), Evaluation (perbandingan metrik), hingga Deployment (formulasi predictor dan dokumentasi hasil).