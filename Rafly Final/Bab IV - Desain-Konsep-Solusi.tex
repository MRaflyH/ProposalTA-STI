% ==========================================
% BAB IV DESAIN KONSEP SOLUSI
% ==========================================
\chapter{DESAIN KONSEP SOLUSI}
\label{chap:desain-konsep-solusi}

Sesuai dengan metodologi yang sudah ditentukan, langkah-langkah pengembangan model prediksi GFD dapat dilihat pada Gambar \ref{gambar:CRISP}. Proses dimulai dengan tahap pamahaman bisnis untuk menetapkan tujuan dan rencana proyek, diikuti oleh pemahaman data guna mengumpulkan serta memvalidasi kualitas data awal. Data tersebut kemudian diproses melalui tahap persiapan data yang mencakup pembersihan dan pemformatan agar siap digunakan dalam tahap pemodelan dengan algoritma diterapkan dan dioptimalkan. Setelah itu, model melalui tahap evaluasi untuk memastikan efektivitasnya dalam menjawab kebutuhan bisnis sebelum akhirnya masuk ke tahap \textit{Deployment} untuk diimplementasikan ke dalam operasional nyata atau disajikan sebagai laporan akhir.

\begin{figure}[h]
	\centering
  \captionsetup{justification=centering}
    	\includegraphics[width=1\textwidth]{image/CRISP.png}
	\caption{Metodologi CRISP-DM}
	\label{gambar:CRISP}
\end{figure}

\begin{figure}[h]
	\centering
  \captionsetup{justification=centering}
    	\includegraphics[width=1\textwidth]{image/Desain.png}
	\caption{Desain Konsep Solusi}
	\label{gambar:Desain}
\end{figure}

Sistem ini dibangun di atas fondasi data meteorologis hibrida yang menggabungkan pengamatan lokal dan global untuk menganalisis karakteristik petir. Sumber data utama terdiri dari \textit{dataset} sambaran petir lokal dari institusi seperti PLN yang mencatat lokasi dan waktu kejadian secara presisi di wilayah tropis Indonesia, serta \textit{dataset} satelit NASA yang menyediakan informasi global mengenai kilatan cahaya (\textit{flashes}) dan GFD, khususnya di wilayah subtropis. Integrasi kedua \textit{dataset} ini memberikan gambaran komprehensif mengenai pola aktivitas petir di berbagai zona iklim yang berbeda.

Sebelum data tersebut dimasukkan ke dalam model, dilakukan serangkaian proses pra-pemrosesan melalui modul pengolahan data. Tahap ini berfokus pada sinkronisasi resolusi spasial dan temporal untuk menyelaraskan sistem \textit{grid} antar-\textit{dataset} serta melakukan kalkulasi nilai GFD per \textit{grid}. Selain itu, dilakukan normalisasi dan \textit{scaling} fitur agar data numerik tersebut kompatibel dengan mekanisme \textit{encoding} kuantum, misalnya dengan menyesuaikan rentang nilai agar dapat direpresentasikan sebagai sudut rotasi dalam sirkuit kuantum.

Inti dari pemrosesan cerdas sistem ini terletak pada modul Qiskit yang mengimplementasikan algoritma QNN. Proses komputasi dimulai dengan penggunaan \textit{feature map}, seperti \textit{angle encoding} atau lainnya, yang bertugas memetakan fitur meteorologis klasik ke dalam \textit{state qubit}. \textit{Output} regresi GFD akhirnya diperoleh yang menghitung \textit{expectation value} dari sirkuit kuantum tersebut.

Untuk memastikan validitas dan keandalan model, sistem dilengkapi dengan modul evaluasi dan analisis yang komprehensif. Modul ini menerapkan fungsi evaluasi terstandardisasi yang memungkinkan evaluasi adil model QNN. Analisis difokuskan pada perbandingan performa di dua domain geografis utama—tropis dan subtropis, serta pengujian skenario \textit{cross-domain} untuk mengukur sejauh mana model mampu mengelaborasi pola petir dari satu wilayah iklim ke wilayah lainnya.

Perbedaan signifikan antara sistem saat ini dan sistem yang diusulkan terletak pada pendekatan integrasi data serta paradigma komputasi yang diterapkan. Sistem yang ada saat ini umumnya masih bergantung pada model numerik, statistik klimatologis, atau ML klasik yang cenderung dikembangkan berdasarkan karakteristik wilayah subtropis, dengan integrasi yang terbatas antara data lokal tropis (PLN) dan data global (NASA). Sebaliknya, sistem yang diusulkan menawarkan solusi QML yang menyatukan kedua sumber data tersebut ke dalam satu \textit{pipeline} analitik terpadu dengan penandaan domain yang jelas. Jika sebelumnya belum ada pemanfaatan teknologi kuantum, sistem baru ini menyediakan jalur pemodelan QNN berbasis VQC.