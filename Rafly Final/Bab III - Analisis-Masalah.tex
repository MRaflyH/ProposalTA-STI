% ============================================================================================
% BAB III ANALISIS MASALAH
% Pembagian subbab tidak rigid dan dapat bervariasi. Bab ini minimal berisi analisis kebutuhan
% fungsional dan nonfungsional, analisis berbagai alternatif solusi yang dapat ditawarkan, dan
% metode pemilihan solusi yang diusulkan.
% ============================================================================================
\chapter{ANALISIS MASALAH}
\label{chap:analisis-masalah}
\section{Analisis Kondisi Saat Ini}

Prediksi GFD dan parameter petir saat ini masih didominasi oleh model numerik dan algoritma ML klasik yang memiliki keterbatasan dalam menangkap proses fisik atmosfer yang kompleks serta sangat bergantung pada \textit{feature engineering} manual. Walaupun pendekatan DL seperti CNN dan LSTM telah digunakan untuk \textit{nowcasting}, model-model ini menuntut biaya komputasi yang tinggi dengan jutaan parameter dan kerap mengalami kesulitan generalisasi pada kejadian ekstrem akibat sifat atmosfer yang \textit{chaotic}. Tantangan lainnya adalah ketidaksesuaian standar proteksi internasional yang berbasis data subtropis bila diterapkan di Indonesia, mengingat wilayah tropis memiliki karakteristik awan CB dan parameter petir yang lebih intensif. Di sisi lain, potensi komputasi kuantum atau QML sejauh ini baru diuji pada variabel cuaca umum seperti suhu dan angin, namun belum dimanfaatkan secara spesifik untuk memodelkan GFD maupun mengintegrasikan data global dan lokal guna mengatasi disparitas iklim tersebut di era NISQ ini.

\section{Analisis Kebutuhan}

\subsection{Identifikasi Masalah Pengguna}

Metode komputasi klasik saat ini memiliki keterbatasan dalam menangani sistem atmosfer yang \textit{chaotic} dan memakan biaya komputasi tinggi akibat kompleksitas parameter fisik yang terlibat. Meskipun teknologi komputasi kuantum telah menunjukkan potensi besar dalam menangani pola nonlinear pada parameter cuaca umum seperti kecepatan angin, penerapannya belum pernah dilakukan secara spesifik untuk memprediksi GFD. Kesenjangan ini menghambat pemanfaatan keunggulan \textit{superposition} dan \textit{entanglement} untuk menyelesaikan masalah prediksi petir yang sulit dijangkau metode konvensional.

Kendala utama bagi pengguna adalah ketiadaan formulasi pemetaan matematis (\textit{mapping}) yang tervalidasi, yang memungkinkan input variabel atmosfer dikonversi langsung menjadi nilai estimasi GFD. Pendekatan saat ini masih sangat bergantung pada ekstraksi fitur manual yang rumit dan sering kali mengabaikan keterkaitan fisika dasar pemicu petir. Hal ini menciptakan kebutuhan mendesak akan suatu fungsi prediktor komputasi yang praktis, dengan parameter cuaca dapat dimasukkan ke dalam sirkuit teroptimasi untuk menghasilkan luaran nilai GFD yang akurat.

Akurasi prediksi semakin sulit dicapai akibat disparitas karakteristik yang signifikan antara petir tropis dan subtropis. Standar proteksi dan model prediksi yang tersedia saat ini umumnya dikembangkan berbasis data subtropis, sehingga menimbulkan bias dan ketidaksesuaian ketika diterapkan di wilayah tropis. Kondisi ini menyulitkan pengguna dalam menghitung GFD yang reliabel karena belum adanya model yang terbukti mampu melakukan generalisasi pada kedua domain iklim yang berbeda tersebut.

\subsection{Kebutuhan Fungsional}

Berdasarkan identifikasi masalah pengguna, kebutuhan fungsional model ditunjukkan pada Tabel \ref{tbl:tabelFungsional}.

\input table/tabelFungsional.tex

\subsection{Kebutuhan Nonfungsional}

Berdasarkan identifikasi masalah pengguna, kebutuhan nonfungsional model ditunjukkan pada Tabel \ref{tbl:tabelNonFungsional}.

\input table/tabelNonFungsional.tex

\section{Analisis Pemilihan Solusi}

\subsection{Alternatif Solusi}

Alternatif dari pemilihan solusi dibagi menjadi pendekatan-pendekatan algoritma QML yang telah dibahas pada studi literatur.

\begin{enumerate}
\item	\textit{Quantum Generative Adversarial Networks} (QGAN)
\item   \textit{Quantum Neural Network} (QNN)
\item   \textit{Quantum Support Vector Regression} (QSVR)
\item   \textit{Quantum $K$ Nearest Neighbors} (QkNN)
\item   \textit{Quantum Decision Tree} (QDT)
\end{enumerate}

\subsection{Analisis Penentuan Solusi}

Penentuan solusi dilakukan berdasarkan parameter penilaian berikut.

\begin{enumerate}
\item	\textit{Suitability for Regression} (GFD): Seberapa alami algoritma tersebut beradaptasi untuk memprediksi nilai numerik kontinu (densitas)? Beberapa algoritma pada dasarnya adalah \textit{classifiers} (pengklasifikasi) dan memerlukan modifikasi yang kompleks untuk dapat menghasilkan nilai \textit{density}.
\item   \textit{Handling High-Dimensional Data} (Fitur Meteorologi): Data meteorologi (Suhu, Kelembapan, Tekanan, CAPE, \textit{Shear}) bisa menjadi sangat kompleks. Seberapa efisien algoritma tersebut dapat memetakan fitur klasik ini ke dalam \textit{Hilbert space} kuantum (efisiensi \textit{Feature Map})?
\item   \textit{Noise Resilience} (Era \textit{NISQ}): Komputer kuantum saat ini memiliki tingkat \textit{noise} yang tinggi. Seberapa baik kinerja algoritma tanpa adanya \textit{error correction} yang sempurna? (Metode \textit{Kernel} sering kali menoleransi \textit{noise} lebih baik daripada \textit{deep circuits}).
\item   \textit{Training Stability & Convergence}: Seberapa mudah algoritma tersebut untuk dilatih? (misalnya, kemampuan menghindari \textit{Barren Plateaus} pada QNN atau \textit{mode collapse} pada QGAN).
\item   \textit{Qiskit Implementation Maturity}: Seberapa siap dan tersedia \textit{library} serta \textit{primitives} dalam ekosistem Qiskit saat ini (khususnya \textit{Qiskit Machine Learning})?
\end{enumerate}

Masing-masing parameter dibobotkan pada Tabel \ref{tbl:tabelKriteriaSolusi}. Penentuan bobot kriteria ini didasarkan pada prioritas utama penelitian, yaitu kemampuan model dalam mengolah kompleksitas data meteorologi. Kriteria \textit{Handling Complex Meteo Features} diberikan bobot dominan sebesar 50\%, karena fenomena petir (GFD) melibatkan interaksi variabel cuaca yang sangat nonlinear dan berdimensi tinggi, sehingga kemampuan \textit{Feature Map} dalam merepresentasikan data tersebut menjadi faktor penentu utama akurasi. Sementara itu, aspek \textit{Noise Resilience} dan \textit{Training Stability} diberikan bobot yang lebih rendah (total 15\%) karena penelitian ini difokuskan pada validasi algoritma menggunakan simulasi (\textit{quantum simulator}) yang ideal, sehingga kendala \textit{noise} perangkat keras (\textit{NISQ}) dapat diminimalisir dan bukan menjadi hambatan utama dalam tahap eksperimen ini.

Masing-masing algoritma dinilai dengan skala penilaian berikut.
\begin{enumerate}
\item	4 - Sangat Baik / Sangat Sesuai (\textit{Excellent}):\\
Algoritma memiliki kemampuan superior dalam memenuhi kriteria, dukungan \textit{library} yang matang, atau karakteristik yang sangat tepat untuk menyelesaikan masalah regresi GFD tanpa modifikasi besar.
\item   3 - Baik / Sesuai (\textit{Good}):\\
Algoritma mampu memenuhi kriteria dengan baik dan dapat diandalkan, namun masih memiliki sedikit keterbatasan teknis atau kinerja dibandingkan dengan skor 4.
\item   2 - Cukup / Kurang Sesuai (\textit{Fair}):\\
Algoritma dapat digunakan namun memerlukan modifikasi signifikan, upaya komputasi yang besar, atau secara teoritis kurang optimal untuk karakteristik data meteorologi.
\item   1 - Kurang / Tidak Sesuai (\textit{Poor}):\\
Algoritma memiliki keterbatasan fundamental untuk kriteria tersebut, sangat sulit diimplementasikan, atau tidak dirancang untuk tujuan prediksi nilai kontinu (regresi).
\end{enumerate}

\input table/tabelKriteriaSolusi.tex

Berdasarkan penilaian pada Tabel \ref{tbl:tabelKriteriaSolusi}, QNN ditetapkan sebagai metode utama yang akan digunakan. QNN mendapatkan skor sempurna (4) pada kemampuan menangani fitur kompleks. Arsitektur VQC pada QNN memungkinkan parameter dalam \textit{feature map} untuk ikut dilatih (\textit{trainable}), sehingga model dapat belajar representasi data cuaca yang paling optimal secara adaptif. Meskipun QNN memiliki kelemahan pada stabilitas pelatihan akibat fenomena \textit{Barren Plateaus} (Skor 2), dampak negatif ini diredam oleh bobot kriteria stabilitas yang kecil (5\%). Dukungan pustaka Qiskit yang matang (Skor 4) juga memastikan implementasi QNN dapat dilakukan secara efisien.