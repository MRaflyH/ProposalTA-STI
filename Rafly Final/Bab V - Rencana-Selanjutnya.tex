% ==========================================
% BAB V RENCANA SELANJUTNYA
% ==========================================
\chapter{RENCANA SELANJUTNYA}
\label{chap:rencana-selanjutnya}

Bab ini merinci strategi eksekusi penelitian selama 16 minggu ke depan untuk mengembangkan dan memvalidasi model prediksi GFD berbasis QNN. Rencana ini disusun secara sistematis mengikuti kerangka kerja CRISP-DM, mulai dari penyiapan data meteorologis yang kompleks hingga evaluasi lintas domain iklim.

\section{Rencana Implementasi}

Pelaksanaan penelitian dibagi ke dalam empat fase bulanan yang berurutan. Fase pertama, yang berlangsung pada minggu ke-1 hingga ke-4, difokuskan sepenuhnya pada persiapan data dan rekayasa fitur. Kegiatan ini dimulai dengan akuisisi data petir dari PLN dan satelit NASA atau data atmosfer lainnya, yang kemudian disinkronisasi ke dalam grid spasial. Mengingat keterbatasan \textit{qubit}, reduksi dimensi dilakukan secara ketat untuk memampatkan puluhan variabel cuaca menjadi fitur utama yang siap dikodekan ke dalam sirkuit kuantum.

Memasuki bulan kedua (minggu ke-5 hingga ke-8), fokus beralih pada pengembangan arsitektur model dan lingkungan komputasi. Pada tahap ini, peneliti akan membangun sirkuit variasional menggunakan \textit{Qiskit}, memilih \textit{feature map} untuk menangkap korelasi nonlinear, dan merancang algoritma yang efisien. Sistem pelatihan hibrida klasik-kuantum akan diimplementasikan untuk tahan terhadap noise, disertai dengan memastikan stabilitas konvergensi model sebelum pelatihan skala penuh dilakukan.

Fase eksperimentasi utama dilaksanakan pada bulan ketiga (minggu ke-9 hingga ke-12). Kegiatan inti pada fase ini adalah melatih dua model terpisah, yaitu model domain tropis menggunakan data Indonesia dan model domain subtropis menggunakan data global. Setelah model terlatih, serangkaian uji silang (\textit{cross-domain testing}) dilakukan dengan menerapkan model tropis pada data subtropis dan sebaliknya. Tujuannya adalah untuk mengukur kesenjangan generalisasi (\textit{generalization gap}) sebagai bukti empiris adanya disparitas karakteristik petir antar-wilayah.

Bulan terakhir (minggu ke-13 hingga ke-16) didedikasikan untuk analisis, interpretasi, dan penyusunan laporan. Peneliti akan melakukan interpretasi fisika terhadap hasil prediksi. Seluruh temuan akan didokumentasikan dalam Tugas Akhir 2, dilanjutkan dengan revisi naskah bersama dosen pembimbing dan persiapan materi presentasi untuk sidang akhir skripsi.

\section{Desain Pengujian dan Evaluasi}

Evaluasi kinerja model dirancang untuk menjawab hipotesis penelitian melalui pendekatan komparatif yang ketat di lingkungan simulator \textit{Qiskit}. Pengujian akan dilaksanakan dalam tiga skenario utama, \textit{benchmarking} model QNN untuk memvalidasi kelayakan teknis, pengujian \textit{in-domain} untuk menetapkan akurasi dasar, dan pengujian \textit{cross-domain}untuk menganalisis penurunan performa saat model diterapkan pada iklim yang berbeda.

Untuk mengukur keberhasilan prediksi secara kuantitatif, penelitian ini menggunakan metrik RMSE sebagai indikator utama deviasi prediksi GFD, dilengkapi dengan koefisien determinasi untuk melihat kemampuan model dalam menjelaskan varians data. Selain angka statistik, analisis kualitatif dilakukan menggunakan visualisasi guna mengidentifikasi apakah model memiliki kelemahan spesifik pada topografi tertentu, seperti wilayah pegunungan atau lautan.

\section{Analisis Risiko dan Mitigasi}

Pengembangan model kuantum pada era NISQ menghadapi risiko teknis utama berupa fenomena \textit{Barren Plateaus}, dengan gradien menghilang sehingga model gagal belajar. Untuk memitigasi hal ini, penelitian akan menggunakan sirkuit kuantum yang dangkal (\textit{shallow circuits}), teknik inisialisasi parameter identitas, serta fungsi biaya lokal yang lebih stabil. Risiko teknis lainnya adalah waktu tunggu antrian yang lama pada perangkat keras IBM Quantum, yang akan diatasi dengan memprioritaskan penggunaan simulator lokal berkinerja tinggi untuk sebagian besar eksperimen dan hanya menggunakan perangkat keras nyata untuk validasi akhir sampel kecil.   

Dari sisi data, terdapat risiko \textit{sparsity} atau banyaknya nilai nol pada data satelit akibat keterbatasan waktu lintas orbit. Mitigasi dilakukan dengan melakukan agregasi temporal data (misalnya menjadi rata-rata bulanan) untuk mengisi kekosongan data observasi. Terakhir, untuk mengantisipasi risiko manajemen waktu akibat kompleksitas pemrograman hibrida, akan diterapkan solusi dengan pipeline sederhana diselesaikan terlebih dahulu sebelum meningkatkan kompleksitas model secara bertahap jika waktu memungkinkan.   