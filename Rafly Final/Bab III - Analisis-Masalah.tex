% ============================================================================================
% BAB III ANALISIS MASALAH
% Pembagian subbab tidak rigid dan dapat bervariasi. Bab ini minimal berisi analisis kebutuhan
% fungsional dan nonfungsional, analisis berbagai alternatif solusi yang dapat ditawarkan, dan
% metode pemilihan solusi yang diusulkan.
% ============================================================================================
\chapter{ANALISIS MASALAH}
\label{chap:analisis-masalah}
\section{Analisis Kondisi Saat Ini}

Prediksi GFD dan parameter petir saat ini didominasi model numerik dan ML klasik

NWP dan parameterisasi petir berbasis rumus fisik masih terbatas dalam menangkap proses multi-skala dan sifat chaotic atmosfer.
Algoritma klasik (DT, RF, SVM, ANN) sangat bergantung pada feature engineering manual dan sering kesulitan memodelkan interaksi non-linear antar variabel meteorologis.

Pendekatan deep learning (CNN/LSTM/ConvLSTM) sudah digunakan untuk nowcasting petir

Arsitektur kompleks (misalnya encoder–decoder 3D CNN, ConvLSTM) dapat memproses data satelit/radar multi-dimensi tetapi memiliki puluhan juta parameter yang mahal secara komputasi.
Biaya pelatihan dan inferensi tinggi membatasi penerapan operasional, terutama jika ingin melakukan eksperimen multi-skenario atau studi generalisasi lintas wilayah.

Keterbatasan generalisasi model klasik dan deep learning

Atmosfer bersifat dinamis dan chaotic; perubahan kecil pada kondisi awal dapat memicu perbedaan besar pada output, sehingga akurasi model klasik mudah turun di luar domain data latih.
Peristiwa ekstrem seperti GFD tinggi relatif jarang, sehingga model cenderung bias ke kondisi normal dan kurang akurat untuk ekor distribusi (tail events).

Konteks Indonesia/tropis vs subtropis belum terakomodasi dengan baik

Indonesia memiliki awan CB yang lebih lebar, long tail wave pada arus petir, dan densitas sambaran ke tanah yang lebih tinggi dibanding wilayah subtropis.
Standar proteksi petir (IEC, NFPA, dll.) banyak dibangun dari data subtropis/lintang menengah, sehingga berpotensi tidak sepenuhnya sesuai jika langsung diterapkan pada iklim tropis maritim Indonesia.

Keterpisahan sumber data global dan lokal

Data satelit global (LIS/OTD) memberikan climatology GFD resolusi global, sedangkan data lokal (mis. PLN) merekam sambaran aktual di jaringan tenaga listrik.
Integrasi dan pemodelan bersama kedua sumber (tropis vs subtropis) untuk menghasilkan model GFD adaptif secara regional masih minim.

Posisi terkini komputasi kuantum dan QML

QML telah dicoba untuk prediksi suhu permukaan, kecepatan angin, iradiasi matahari, dan GHI dengan hasil yang kompetitif, namun belum ada studi spesifik untuk GFD petir.
Perangkat masih berada di era NISQ, sehingga banyak penelitian bergantung pada simulator dan perangkat IBM Quantum ukuran kecil dengan sirkuit dangkal.

\section{Analisis Kebutuhan}
\dots

\subsection{Identifikasi Masalah Pengguna}
Insinyur proteksi petir dan perencana sistem tenaga (mis. PLN)

Membutuhkan estimasi GFD yang reliabel per lokasi untuk menghitung risiko dan merancang sistem proteksi (tower, gardu, jaringan transmisi/distribusi) secara ekonomis namun aman.
Menghadapi ketidakpastian karena peta GFD berbasis subtropis mungkin tidak cocok dengan karakteristik petir tropis Indonesia.

Peneliti meteorologi dan klimatologi petir

Memerlukan model yang mampu mengeksplorasi hubungan non-linear antara variabel meteorologis (CAPE, presipitasi, kelembapan, dll.) dengan GFD, termasuk perbedaan tropis vs subtropis.
Kesulitan mengevaluasi apakah model yang terlatih di suatu rezim iklim dapat digeneralisasi ke rezim lain tanpa kehilangan akurasi secara drastis.

Penyusun standar dan regulator

Perlu evidensi ilmiah berbasis data untuk meninjau kecocokan standar proteksi petir yang bersumber dari data subtropis ketika diterapkan di Indonesia dan wilayah tropis lainnya.
Memerlukan indikator kuantitatif yang menunjukkan gap antara GFD “standar” vs GFD aktual tropis.

Komunitas QML dan komputasi kuantum terapan

Membutuhkan studi kasus konkret di domein meteorologi tropis untuk mengevaluasi benefit praktis QNN/VQC dibanding model klasik.
Menghadapi keterbatasan contoh implementasi end-to-end yang menghubungkan data cuaca/petir dunia nyata dengan arsitektur QNN di Qiskit.

Masalah praktis lintas pemangku kepentingan

Ketidakakuratan GFD mengarah pada desain proteksi over-designed (mahal) atau under-designed (tidak aman).
Keterbatasan model yang tidak generalizable menyulitkan transfer pengetahuan dari wilayah subtropis ke tropis dan sebaliknya.

\subsection{Kebutuhan Fungsional}
Model prediksi GFD kontinu berbasis QNN/VQC

Menerima input berupa fitur meteorologis dan/atau klimatologis yang relevan (mis. CAPE, presipitasi, suhu permukaan, kelembapan, variabel geopotensial, dsb.).
Menghasilkan output berupa estimasi nilai GFD (mis. sambaran per km² per tahun) untuk suatu grid/wilayah.

Kemampuan membedakan dan memodelkan pola tropis vs subtropis

Mengakomodasi penandaan domain (tropis/subtropis) atau koordinat geografis sebagai bagian dari fitur.
Mampu mempelajari perbedaan pola GFD pada kedua rezim iklim dan memvisualisasikannya melalui analisis error dan distribusi prediksi.

Dukungan eksperimen generalisasi lintas domain

Memungkinkan skenario train-on-tropics, test-on-subtropics dan sebaliknya untuk mengukur kemampuan adaptasi model.
Menyediakan pipeline untuk membandingkan performa cross-domain vs in-domain menggunakan metrik yang sama.

Integrasi dengan baseline klasik

Menyertakan implementasi model klasik (mis. RF, SVR, MLP sederhana, atau DL ringan) pada dataset yang sama sebagai pembanding kinerja.
Menyediakan fungsi evaluasi bersama (common evaluation interface) antara QNN dan baseline klasik.

Manajemen eksperimen dan konfigurasi

Menyimpan konfigurasi feature map, ansatz, optimizer, dan hyperparameter lain dalam bentuk yang dapat direplikasi.
Menyediakan fungsi untuk menyimpan dan memanggil kembali model/predictor yang telah terlatih untuk keperluan analisis dan validasi lanjutan.

\subsection{Kebutuhan Nonfungsional}
Keterbatasan komputasi dan hardware

Implementasi utama menggunakan simulator Qiskit pada komputer pribadi, sehingga jumlah qubit dan kedalaman sirkuit harus dibatasi.
Waktu pelatihan dan inferensi harus tetap dalam rentang yang wajar untuk ukuran dataset yang tersedia, sehingga arsitektur QNN perlu efisien.

Kualitas dan kelengkapan dataset

Data PLN dan NASA/LIS-OTD mungkin memiliki cakupan temporal/spasial yang berbeda, missing values, dan ketidakseimbangan distribusi GFD.
Diperlukan prosedur preprocessing dan pembersihan data yang jelas untuk mengurangi bias dan noise.

Interpretabilitas dan transparansi minimal

Walaupun QNN bersifat black-box, hasil perlu dilengkapi dengan analisis metrik (MSE, RMSE, MAE, R²) dan visualisasi sederhana (mis. plot aktual vs prediksi, sebar error).
Dokumentasi pemilihan fitur dan justifikasi desain arsitektur (feature map, ansatz) harus jelas.

Reproducibility dan maintainability

Kode, konfigurasi eksperimen, dan versi pustaka (Python, Qiskit) perlu dicatat untuk mendukung replikasi hasil di masa depan.
Struktur proyek (notebook, modul Python) harus rapi untuk memudahkan pengembangan lanjutan oleh peneliti lain.

Kesesuaian dengan batasan era NISQ

Arsitektur QNN harus mempertimbangkan noise dan batas kedalaman sirkuit jika kelak dijalankan di perangkat IBM Quantum nyata.
Desain perlu kompatibel dengan teknik error mitigation sederhana yang tersedia di Qiskit.

\section{Analisis Pemilihan Solusi}
\subsection{Alternatif Solusi}
Pendekatan ML klasik murni

Menggunakan RF, SVR, gradient boosting, atau MLP untuk memprediksi GFD dari fitur meteorologis.
Kelebihan: tooling matang, implementasi sederhana, kebutuhan komputasi diketahui.

Pendekatan deep learning spatiotemporal

Mengadopsi CNN, LSTM, atau ConvLSTM yang telah digunakan di penelitian nowcasting petir dan cuaca.
Kelebihan: mampu menangkap pola spasial-temporal kompleks, tetapi memerlukan data besar dan komputasi berat.

Hybrid klasik dengan feature engineering intensif

Menggabungkan transformasi fitur (mis. PCA, statistik klimatologis) dengan model klasik (RF/SVR).
Mengandalkan kreativitas feature design untuk menangkap non-linearitas dan perbedaan tropis vs subtropis.

QML berbasis QSVR (quantum kernel method)

Menggunakan QSVR dengan quantum feature map untuk membangun kernel non-linear di ruang Hilbert.
Potensial unggul dalam time-series forecasting, namun training membutuhkan komputasi matriks Gram M×M yang mahal untuk M sampel besar.

QML berbasis QNN/VQC

Menggunakan arsitektur VQC dengan EstimatorQNN sebagai regressor GFD.
Memanfaatkan hybrid loop klasik-kuantum untuk melatih parameter ansatz secara iteratif.

\subsection{Analisis Penentuan Solusi}
Kesesuaian dengan tujuan penelitian

Tujuan utama adalah mengeksplorasi potensi QML, khususnya QNN/VQC di Qiskit, untuk prediksi GFD tropis dan subtropis, bukan sekadar memaksimalkan akurasi dengan model klasik.
QNN menyediakan kerangka yang lebih kaya untuk mengkaji expressibility dan effective dimension dibanding QSVR semata.

Pertimbangan kompleksitas komputasi

QSVR berbasis kernel memerlukan perhitungan Gram matrix berukuran M×M yang berskala minimal O(M²–M³), sehingga kurang efisien untuk jumlah sampel latih menengah.
Training QNN/VQC dengan optimizer berbasis gradient atau heuristic cenderung menskala linier terhadap jumlah sampel, lebih sesuai untuk eksperimen pada dataset GFD ukuran menengah.

Kesesuaian dengan era NISQ dan ekosistem Qiskit

VQC dirancang untuk perangkat NISQ dengan kedalaman sirkuit yang dapat dikontrol, serta mudah dioptimasi dengan classical optimizer.
Qiskit menyediakan primitive Estimator, library feature map (mis. ZZFeatureMap), dan ansatz (TwoLocal, Ry, dsb.) yang langsung dapat dimanfaatkan.

Fleksibilitas arsitektur dan desain eksperimen

QNN memungkinkan eksplorasi berbagai kombinasi feature map–ansatz dan kedalaman sirkuit untuk menyesuaikan dengan kompleksitas data dan keterbatasan komputasi.
Arsitektur yang sama dapat diadaptasi untuk studi generalisasi tropis vs subtropis dengan mengganti skema input atau konfigurasi training.

Nilai tambah ilmiah dibanding baseline klasik

Dengan tetap menyertakan baseline klasik, pemilihan QNN/VQC memberikan kontribusi ilmiah berupa perbandingan sistematis antara dua paradigma komputasi pada kasus GFD.
Hasil eksperimen dapat menjadi acuan awal bagi pengembangan standar dan aplikasi QML lebih lanjut di bidang meteorologi tropis.

Peluang implementasi di hardware kuantum nyata

Desain VQC yang dangkal dan hemat qubit membuka peluang untuk uji coba di perangkat IBM Quantum ketika sumber daya tersedia.
Hal ini memberikan nilai demonstrasi praktis yang tidak dapat diperoleh jika hanya mengandalkan model klasik.